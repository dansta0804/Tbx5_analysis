\documentclass[12pt]{article}


%\documentclass[17pt]{extarticle}

%\usepackage{extsizes}
\usepackage{indentfirst}
\usepackage[utf8x]{inputenc}
\usepackage[T1]{fontenc}
\usepackage[english,lithuanian]{babel}
\usepackage{array}
\usepackage{caption}
\usepackage{makecell}
\usepackage[euler]{textgreek}
\usepackage{hyperref}
\usepackage{multirow}
\usepackage{boldline}
\usepackage{floatrow}
\floatsetup[table]{capposition=top}

\usepackage{amsmath, amsthm, amssymb}
\usepackage{graphicx}
\usepackage{setspace}
\usepackage{verbatim}
\usepackage[left=3cm,top=2cm,right=1.5cm,bottom=2cm]{geometry}
\usepackage{floatrow}
\newfloatcommand{capbtabbox}{table}[][\FBwidth]
\usepackage{blindtext}

\onehalfspacing

\newcommand{\EE}{\mathbb{E}\,} % Mean
\newcommand{\ee}{{\mathrm e}}  % nice exponent
\newcommand{\dd}{{\mathrm d}}
\newcommand{\RR}{\mathbb{R}}

\begin{document}
\selectlanguage{lithuanian}

\begin{titlepage}
\vskip 20pt
\begin{center}
\includegraphics[scale=0.5]{MIF}
\end{center}

%%%%%%%%%%%%%%%%%%%%%%%%%%%%%%%
% TITULINIO PUSLAPIO TEKSTAS
%%%%%%%%%%%%%%%%%%%%%%%%%%%%%%%

\vskip 20pt
\centerline{\bf \large \textbf{VILNIAUS UNIVERSITETAS}}
\bigskip
\centerline{\large \textbf{MATEMATIKOS IR INFORMATIKOS FAKULTETAS}}
\bigskip
\centerline{\large \textbf{BIOINFORMATIKOS BAKALAURO STUDIJŲ PROGRAMA}}

\vskip 90pt
\begin{center}
    {\bf \LARGE Tbx5 transkripcijos faktoriaus tyrimas \emph{Mus musculus}
     širdies ląstelėse}
\end{center}
\begin{center}
    {\bf \Large Research of Tbx5 transcription factor in \emph{Mus musculus}
     heart cells}
\end{center}
\vskip 20pt
\centerline{\bf \large \textbf{Kursinis darbas}}
\bigskip
\vskip 40pt

\hskip 140pt {\large Autorius: Danielė Stasiūnaitė}

\hskip 140pt{\large VU el. p.: (daniele.stasiunaite@mif.stud.vu.lt)}
\bigskip
\vskip 20pt

\hskip 140pt {\large Darbo vadovas: J. m. d. Kotryna Kvederavičiūtė}
\vskip 60pt
\vskip 40pt
\centerline{\large \textbf{Vilnius}}
\centerline{\large \textbf{2022}}
\newpage
\end{titlepage}

\selectlanguage{lithuanian}

%%%%%%%%%%%%%%%%%%%%%
% TURINIO PUSLAPIS
%%%%%%%%%%%%%%%%%%%%% 
\tableofcontents
\newpage

%%%%%%%%%%%%%%%%%%%%%%%%%%%%%%%%%%%%
% LIETUVIŠKOS SANTRAUKOS PUSLAPIS
%%%%%%%%%%%%%%%%%%%%%%%%%%%%%%%%%%%%

\section*{Santrauka}
Regeneracijos procesų tyrinėjimas yra svarbi sritis, galinti prisidėti prie
įvairių ligų bei traumų gydymo, todėl yra svarbu išsiaiškinti šį procesą
valdančius mechanizmus bei juose dalyvaujančius įvairių genų produktus -
baltymus.
                                                                                     
Šiame darbe naudojantis R programavimo kalbos bibliotekomis bei
bioinformatiniais komandinės eilutės bei internetiniai duomenų apdorojimo
bei analizės įrankiais išanalizuota, kokiuose ChIP sekoskaitos metodu gautų
naminės pelės (lot. \emph{Mus musculus}) ląstelių regionų mėginiuose Tbx5
transkripcijos faktoriaus motyvų yra daugiausiai bei kokie veiksniai gali
įtakoti skirtingą faktoriaus motyvų skaičių naminių pelių genomo sekose.

Atliktos analizės metu mėginiuose identifikuota daug skirtingų motyvų.
Daugiausiai tyrinėjamo Tbx5 transkripcijos faktoriaus motyvų identifikuota
naminių pelių embrionų fibroblastų ląstelių, veiktų serino/treonino kinaze 1
(Akt1) bei kardiogeniniais transkripcijos faktoriais GATA4, HAND2 ir MEF2C
antroje chromosomoje. Mažiausias Tbx5 transkripcijos faktoriaus motyvų skaičius
būdingas fibroblastams, kurie buvo veikti ne pilnu kardiogeninių faktorių
rinkiniu.

\hfill \break
\textbf{Raktiniai žodžiai:} ChIP sekoskaita, transkripcijos faktorius, Tbx5,
        regionas, motyvas, R.
\newpage

%%%%%%%%%%%%%%%%%%%%%%%%%%%%%%%%%%%%%%%%
% ANGLIŠKOS SANTRAUKOS PUSLAPIS
%%%%%%%%%%%%%%%%%%%%%%%%%%%%%%%%%%%%%%%%

\section*{Summary}
The investigation of regeneration processes is an important field of research
that plays a significant role in the treatment of various diseases and injuries.
Therefore, it is mandatory to determine and apprehend the mechanisms and gene
products that regulate regeneration processes.

In this work, the analysis of the ChIP sequencing peaks samples that were
retrieved from different house mouse (lat. \emph{Mus musculus}) cells was
conducted using functions from R programming language libraries and performing
analysis steps using bioinformatics command-line tools and online services
in order to determine what type of house mouse cells has the greatest number
of Tbx5 transcription factor motifs and what factors might have an impact on
total Tbx5 motif hit count differences among house mouse genome sequences.

The undertaken analysis identified an ample number of motifs. An abundance
of key interest Tbx5 transcription factor was identified in the second
chromosome of house mouse embryonic fibroblasts that were treated with
serine/threonine kinase 1 (Akt1) and GATA4, HAND2, MEF2C cardiogenic
transcription factors. The least count of Tbx5 transcription factor motifs
is common for fibroblasts that were not treated with complete cardiogenic
transcription factor set. 

\hfill \break
\textbf{Keywords:} ChIP-seq, transcription factor, Tbx5, peak, motif, R.
\newpage

%%%%%%%%%%%%%%%%%%%
% ĮVADO PUSLAPIS
%%%%%%%%%%%%%%%%%%%
\section{Įvadas}
\subsection*{Darbo temos aktualumas}

Spartėjanti mokslo raida, įvairūs atradimai bei išradimai stipriai paspartino
ir pagerino įvairių ligų diagnostikos bei prevencijos tyrimus, praplėtė žinias
ląstelės biologijos, genetikos, fiziologijos ir kitose srityse. Viena iš
medicinos sričių, kuri pradėta tyrinėti dar XVIII a., kai buvo nustatyta, kad
kai kurie organizmai geba atsiauginti prarastas arba sužeistas galūnes bei
kitus kūno audinius\cite{REGENERATION}, yra organizmų audinių bei organų
regeneracija.

Šiais laikais regeneracijos tyrimai atliekami su hidromis, planarijomis,
tritonais bei zebražuvėmis\cite{ORGANISMS}, siekiant išsiaiškinti šių organizmų
audinių regeneracijos mechanizmus bei pritaikyti žmonėms, patyrusiems traumas
ar turintiems specifinių kūno audinių pažeidimų. Regeneracijos procese
pagrindinę funkciją atlieka kamieninės ląstelės bei įvairūs transkripcijos
reguliavimo faktoriai, gebantys prisijungti prie DNR chromatino ir skatinti
arba slopinti specifinių genų transkripciją.

Sėkmingam regeneracijos procesų mechanizmų supratimui būtina išsiaiškinti, 
kokie transkripcijos faktoriai dalyvauja šiame procese bei kokias funkcijas jie
atlieka.

\subsection*{Darbo tikslas}

Pagrindinis šio darbo tikslas yra palyginti Tbx5 transkripcijos faktoriaus
regionų skirtumus skirtinguose \emph{Mus musculus} ląstelių mėginiuose,
gautuose panaudojus ChIP-seq metodą.

\subsection*{Uždaviniai}
\begin{itemize}
    \item Nustačius Tbx5 transkripcijos faktoriaus regionų skaičių mėginiuose
    įvertinti, kuriuose mėginiuose praturtintų genominių regionų buvo
    didžiausias ir mažiausias.
    \item Išsiaiškinti, kaip skiriasi regionų skaičius skirtingose genominėse
    pozicijose (chromosomose).
    \item Apskaičiavus tarp mėginių persidengiančių regionų procentinę dalį
    nustatyti, kurie mėginiai yra panašiausi.
    \item Identifikavus Tbx5 transkripcijos faktoriaus sekos motyvą nustatyti,
    kurių mėginių ląstelės turi didžiausią transkripcijos faktoriaus motyvo
    kiekį.
    \item Atlikus \emph{de novo} motyvų paiešką nustatyti, kokios biologinės
    funkcijos yra būdingos identifikuotiems motyvams.
\end{itemize}

\newpage

%%%%%%%%%%%%%%%%%%%%%
% DUOMENŲ APŽVALGA
%%%%%%%%%%%%%%%%%%%%%
\section{Duomenų bazės ir duomenys}
\subsection{GTRD duomenų bazė}
Tyrimui naudoti duomenys atsisiųsti iš GTRD (Gene Transcription Regulation
Database)\cite{GTRD} duomenų bazės, saugančios informaciją apie transkripcijos
sekų ir atviro chromatino regionus. Taip pat duomenų bazėje saugomi
nekartografuojamų regionų duomenys bei potencialūs žmonių bei naminių pelių
regionai, prie kurių gali jungtis transkripcijos faktoriai.

Ši duomenų bazė pasirinkta dėl sistemiškai surinktų ChIP-seq eksperimentų,
kurių metu gauti rezultatai yra unifikuotai apdoroti ir paruošti tyrėjų
meta-analizėms.

GTRD duomenų bazėje duomenys saugomi binariniu anotacijų formatu \emph{bigBed},
leidžiančiu atvaizduoti pasirinktą chromosomos regioną interaktyviose genominės
informacijos vizualizavimo naršyklėse (pavyzdžiui, UCSC Genome
Browser\cite{UCSCGB}) efektyviau nei tekstinis BED formatas.

\subsection{Pasirinktų mėginių charakteristika}
Analizė atlikta, naudojantis 4 nepriklausomais eksperimentais, kuriuos iš viso
sudarė 7 biologinės replikos. Pirmoje lentelėje pateikta informacija apie
tyrimui atlikti naudotus duomenis, surinktus iš naminės pelės (lot. \emph{Mus
musculus}) ląstelių.

\begin{table}[htb]
    \newcolumntype{M}[1]{>{\centering\arraybackslash}m{#1}}
    \small
    \caption*{\small\textbf{1 lentelė. Mėginių charakteristikos}}
    \begin{tabular}{|c|c|c|c|c|c|c|}
    \hline
    %\thead{Sample\\ window}
    \textbf{GTRD ID} & \textbf{Ląstelių tipas} &
        \textbf{\thead{Kamienas}} & \textbf{\thead{Poveikis}} &
        \textbf{Antikūnai} & \textbf{PubMed ID}\\
    \hline
    EXP030898 & \thead{HL - 1\\ (širdies raumens)} & C57BL/6J &
                \thead{TRE\\ promotorius (2 d.)} & - & 21415370\cite{ARTCL1}\\ 
    \hline
    EXP058852 & Širdies prieširdžių & C57BL/6 & - &
                \thead{Tbx5\\ (sc-17866)} & 31080136\cite{ARTCL2}\\
    \hline
    EXP062056 & \thead{Pelių naujagimių širdies\\ fibroblastų, 
                ekspresuojančių\\ didelį kiekį T antigeno, linija} & CD1 &
                \thead{sb431542,\\ xav939} & \thead{anti-TBX5\\ (sc-17866x)} &
                31271750\cite{ARTCL3}\\
    \hline
    EXP058843 & \thead{MEF\\ (embrionų fibroblastai)} & C57BL/6 &
                AGHMT (2 d.) & \thead{anti-Tbx5\\ (sc-17866)} &
                31080136\cite{ARTCL2}\\
    \hline
    EXP058847 & \thead{MEF\\ (embrionų fibroblastai)} & C57BL/6 & GHMT (2 d.) &
                \thead{Tbx5\\ (sc-17866)} & 31080136\cite{ARTCL2}\\
    \hline
    EXP058850 & \thead{MEF\\ (embrionų fibroblastai)} & C57BL/6 & GMT (2 d.) &
                \thead{Tbx5\\ (sc-17866)} & 31080136\cite{ARTCL2}\\
    \hline
    EXP058856 & \thead{MEF\\ (embrionų fibroblastai)} & C57BL/6 &
                \thead{vienas\\ faktorius (2 d.)} & \thead{Tbx5\\ (sc-17866)} &
                31080136\cite{ARTCL2}\\
    \hline
    \end{tabular}
\end{table}

\newpage

\subsection{Santrumpų bei pavadinimų paaiškinimai}
\begin{itemize}
    \item \textbf{HL - 1}: pelių širdies raumens ląstelės, išgautos iš
        navikinių prieširdžių kardiomiocitų linijos. Šios ląstelės gali
        betarpiškai dalintis ir spontaniškai keisti savo formą, vykstant
        širdies raumens susitraukimo/ atsipalaidavimo procesams.
    \item \textbf{MEF}: pelių embrionų fibroblastai (angl. \emph{Mouse
        Embryonic Fibroblast}). Šiai ląstelių linijai būdingas ląstelių
        gyvybingumo apribojimas, reiškiantis, jog šios ląstelės greitai
        pasensta ir miršta.
    \item \textbf{C57BL/6}: inbrydingo (angl. \emph{inbreeding}) būdu išvestų
        naminių pelių veislė. Šios veislės pelėms būdingas itin tamsus
        kailis, padidėjęs jautrumas garsams, kvapams, skausmui ir žemai
        temperatūrai. Ši veislė dažnai naudojama nutukimą ir imuninę sistemą
        tiriančiuose tyrimuose.
    \item \textbf{C57BL/6J}: prie naminių pelių veislės pavadinimo pridėtos
        raidės patikslina, kurioje laboratorijoje veislės išvestos. 'J' raidė
        nurodo, kad pelių veislė išvesta Meino valstijoje (JAV) įsikūrusioje
        Džeksono laboratorijoje\cite{JCKSLAB}.
    \item \textbf{CD1}: autbrydingo (angl. \emph{outbreeding}) būdu išvestų
        naminių pelių veislė. Šios veislės pelėms būdingas baltas kailis.
        Taip pat CD1 pelės dažnai naudojamos genetiniuose, toksikologiniuose,
        farmakologiniuose ir senėjimo tyrimuose.
    \item \textbf{TRE}: tetraciklino atsako elementas (angl. \emph{Tetracycline
        Response Element}). Tai yra 7 DNR sekos fragmentai,
        sudaryti iš 19 nukleotidų ir atskirti trumpesniais sekų fragmentais.
    \item \textbf{sb431542}: stipriai veikianti, selektyvi cheminė medžiaga;
        transformuojančio augimo faktoriaus {\textbeta} (TGF-{\textbeta})
        inhibitorius.
    \item \textbf{xav939}: stipriai veikianti cheminė medžiaga; tankirazės
        inhibitorius. Tankirazė slopina TERF1 baltymo, stabdančio
        telomerazės veiklą, jungimąsi prie telomerinių DNR sekų.
    \item \textbf{AGHMT}: AKT1 - serino/treonino kinazė 1; GATA4, HAND2, MEF2C,
        Tbx5 - kardiogeniniai transkripcijos faktoriai.
    \item \textbf{GHMT}: GATA4, HAND2, MEF2C, TBX5 transkripcijos faktorių
        komplektas.
    \item \textbf{GMT}: GATA4, MEF2C, Tbx5 transkripcijos faktorių komplektas.
    \item \textbf{sc-17866x/ sc-17866}: iš ožkų išskirti antikūnai,
        atpažįstantys žmonių, pelių ir žiurkių Tbx5 antigeną.
\end{itemize}

\subsection{Pasirinktų eksperimentų apžvalga}
\begin{itemize}
    \item \textbf{EXP030898}: vienas mėginys iš septyniolikos eksperimento
        metu tirtų mėginių. Eksperimente buvo siekiama patvirtinti arba
        atmesti hipotezę apie širdies stipriklių (angl. \emph{enhancer})
        identifikiavimą prie chromatino jungiantis keliems transkripcijos
        faktoriams. \\
        HL - 1 širdies raumens ląstelės buvo infekuotos su adenovirusu,
        ekspresuojančiu troponiną T, kuris skatina \emph{rtTA} ir \emph{BirA}
        genų ekspresiją, bei TRE promotoriumi, skatinančiu Tbx5 transkripcijos
        faktoriaus geno raišką. Sąlygos taikytos 48 valandas.
    \item \textbf{EXP058852}: prieširdžių ląstelės buvo du kartus po 24
        valandas laikytos mišinyje su retrovirusais. Papildomi poveikiai nebuvo
        taikyti.
    \item \textbf{EXP062056}: eksperimente ląstelės buvo infekuotos su GATA4,
        Mef2c, ir Tbx5 transkripcijos faktorius sintetinančiais retrovirusais.
        Ląstelės augintos terpėje, kurioje buvo {Tgf\textbeta} inhibitoriaus
        sb431542, skatinančio kardiomiocitų diferenciaciją iš pliuripotentinių
        kamieninių ląstelių, ir Wnt inhibitoriaus xav939, stabdančio
        nediferencijuotų ląstelių sintezę ir skatinančio progenitorinių
        ląstelių kardiomiogenezę.
\end{itemize}

Pasirinktų duomenų rinkinyje naudoti vieno eksperimento, kuriame buvo tirti
širdies ląstelių atsinaujinimo ir diferenciacijos mechanizmai, keturiais
skirtingais poveikiais tirti mėginiai:

\begin{itemize}
    \item \textbf{EXP058843}: embrionų fibroblastai veikti AGHMT. Esant
        kardiogeniniams transkripcijos faktoriams, AKT1 skatina fibroblastų
        diferenciaciją į širdies ląsteles - kardiomiocitus.
    \item \textbf{EXP058847}: neįtraukta AKT1 serino/treonino kinazė 1.
    \item \textbf{EXP058850}: į kardiogeninių transkripcijos faktorių mišinį
        neįtrauktas HAND2 transkripcijos faktorius.
    \item \textbf{EXP058856}: ląstelės veiktos tik vienu faktoriumi, kuris
        straipsnyje nebuvo specifikuotas.
  \end{itemize}
\newpage

%%%%%%%%%%%%
% METODAI
%%%%%%%%%%%%

\section{Tyrimo metodai}
Tbx5 transkripcijos faktoriaus regionų tyrimo analizė atlikta su R programavimo
kalba\cite{R} (4.2.0 versija).

Tarpiniams analizės rezultatams pateikti naudotas komandinės eilutės įrankis
Scikick\cite{SCIK} (0.2.0 versija), leidžiantis generuoti R Markdown (Rmd)
ataskaitas \emph{html} formatu bei kurti struktūrizuotus puslapius, apjungiant
iš daugelio Rmd failų gautus HTML ataskaitų failus.

\subsection{Regionų skaičiaus nustatymas mėginiuose}
\emph{Tbx5} regionų skaičius skirtinguose mėginiuose apskaičiuotas su
standartine R ilgio funkcija \emph{length()}, kuri pritaikyta \emph{GRanges}
objektui, aprašančiam genomines pozicijas bei su jomis susijusias anotacijas.
Objektas sukurtas su \emph{rtracklayer}\cite{R_TRACK} bibliotekos funkcija
\emph{import()}.

Regionų skaičių mėginiuose atvaizduojanti stulpelinė diagrama sukurta su
\emph{ggplot2}\cite{R_GGPLOT} bibliotekos \emph{geom\_bar()} funkcija.

\subsection{Regionų skaičiaus nustatymas chromosomose}
Transkripcijos faktoriaus regionų skaičius skirtingose chromosomose kiekvienam
mėginiui apskaičiuotas, naudojantis standartine R funkcija \emph{length()},
pritaikyta atskiroms chromosomoms, kurių pozicijos aprašytos \emph{GRanges}
objekte. Kiekvieno mėginio Tbx5 transkripcijos faktoriaus pasiskirstymas
chromosomose atvaizduotas su \emph{ggplot()} ir papildoma funkcija
\emph{facet\_wrap()}, sukuriančia atskirus grafikus pagal pasirinktą elementą -
chromosomas.

\subsection{Persidengiančių regionų procentinė dalis}
Persidengiančių regionų tarp mėginių procentinė dalis nustatyta su modifikuota
\emph{Jaccard()} funkcija, apskaičiuojančia, kiek yra sutampančių regionų tarp
dviejų mėginių poros. Naudojantis nemodifikuota funkcija, Jaccard koeficientas
apskaičiuojamas pagal išraišką:

                \[ J(A, B) =  \frac{|A \cap B|}{|A \cup B|} \]

\emph{Jaccard} koeficientas gaunamas iš rinkiniams A ir B bendrų duomenų ilgio
padalinus dviejų duomenų rinkinių bendrą duomenų ilgį.

Modifikavus \emph{Jaccard} koeficiento gavimo funkciją, koeficientas
apskaičiuojamas pagal išraišką, kur sutampančių A ir B rinkinių duomenų ilgis
padalinamas iš A rinkinio ilgio:

                    \[ J(A, B) = \frac{|A \cap B|}{|A|} \]

\emph{Jaccard} koeficiento skaičiavimo funkcija modifikuota, nes skaičiuojant
koeficientą su standartine \emph{Jaccard} funkcija, gaunamas itin didelis
regionų sąjungos skaičius, o persidengiančių regionų skaičius gaunamas mažas,
todėl persidengiančių regionų skaičių padalinus iš regionų sąjungos gaunamas
itin mažas koeficientas, kurio apskaičiuota procentinė dalis neretai neviršijo
1\%.

Tam, jog būtų galima patikimiau įvertinti, kokia pirmojo mėginio procentinė
regionų dalis persidengia su antruoju mėginiu, persidengiančių regionų skaičius
padalintas iš pirmojo mėginio regionų skaičiaus.

Gauti rezultatai atvaizduoti spalvų intensyvumo grafike (angl. \emph{heatmap}),
sukurtame su \emph{ggplot()} ir papildoma funkcija \emph{geom\_tile()}.

\subsection{Tbx5 motyvo nustatymas}
Šiame etape genomines pozicijas aprašantys \emph{bigBed} formato failai
konvertuoti į BED formato failus, pasinaudojus UCSC komandinės eilutės programa
\emph{bigBedToBed}\cite {BBTOBED}.

Sugeneruoti BED formato failai panaudoti pikus atitinkančių sekų iš naminės
pelės genomo gavimui FASTA formatu. Sekos iš genomo išgautos, pasinaudojus
komandinės eilutės įrankio BEDTools\cite{BEDTOOLS} (2.30.0 versija) programa
\emph{getfasta}\cite{GET_FASTA}.

Kiekviename mėginyje esančio Tbx5 transkripcijos faktoriaus motyvo procentinė
dalis apskaičiuota susumavus \emph{Biostrings}\cite{BIOSTR} bibliotekos
funkcijos \emph{countPWM()} rezultatus. Funkcijai \emph{countPWM()} kaip
argumentas pateikta Tbx5 transkripcijos faktoriaus pozicinė svorių matrica bei
mėginio nukleotidų sekų rinkinys FASTA formatu. Gauta funkcijos reikšmė
padalinta iš bendro regionų skaičiaus.

Pozicinė svorių matrica atsisiųsta iš HOCOMOCO\cite{HOCOMOCO} (11.0 versija)
(angl. \emph{HOmo sapiens COmprehensive MOdel COllection}) duomenų bazės
\emph{Homo sapiens} ir \emph{Mus musculus} organizmų transkripcijos faktorių
kolekcijos. Pozicinę svorių matricą atitinkantis sekos logotipas vaizduojamas
pirmame paveiksle (1 pav.).

\begin{figure}[htb]
    \begin{center}
        \includegraphics[width=0.5\linewidth]{../Figures/tbx5_motif.png}
        \caption*{\small\textbf{1 pav. Tbx5 transkripcijos faktoriaus
                                sekos logotipas}}
    \end{center}
\end{figure}

Identifikuotų Tbx5 transkripcijos faktoriaus motyvų skaičius vizualizuotas su
pagrindinėmis \emph{ggplot()} ir \emph{geom\_bar()} funkcijomis.

\subsection{Motyvų paieška \emph{de novo}}
Praturtintų sekų radimui panaudota komandinės eilutės įrankio HOMER\cite{HOMER}
(v4.11 versija) programa \emph{findMotifsGenome.pl}, analizuojanti BED formato
failus (faile specifikuotas pozicijas), ir ieškanti praturtintų sekų atitikimo
anotuotame naminės pelės \emph{mm10} referentiniame genome. Tarp mėginių
persidengiantys motyvai nustatyti, naudojantis R biblioteka
\emph{UpSetR}\cite{UPSETR}.

\subsection{Praturtintų sekų biologinių funkcijų nustatymas}
Identifikuotų motyvų biologinės funkcijos nustatytos, pasinaudojus
UniProt\cite{UNIPROT} duomenų bazės genų ontologijos (angl. \emph{Gene
Ontology (GO)}) biologinių procesų, ląstelinių komponentų ir molekulinių
funkcijų klasifikacija.

\newpage

%%%%%%%%%%%%%%%%%%%%%%%%%%%%%
% GAUTŲ REZULTATŲ APŽVALGA
%%%%%%%%%%%%%%%%%%%%%%%%%%%%%

%%%%%%%%%%%%%%%%%%%%%%%%%%%%%%%%
% REGIONŲ SKAIČIUS MĖGINIUOSE
%%%%%%%%%%%%%%%%%%%%%%%%%%%%%%%%

\section{Rezultatai ir jų aptarimas}
\subsection{Regionų skaičiaus skirtumai tarp mėginių}
Pirmajame analizės etape kiekviename mėginyje nustatytas bendras regionų
skaičius pavaizduotas pirmoje stulpelinėje diagramoje (2 pav.).

\begin{figure}[htb]
    \begin{center}
        \includegraphics[width=0.6\linewidth]{../Figures/total_peak_counts.png}
        \caption*{\small\textbf{2 pav. Regionų skaičių mėginiuose vaizduojanti
                                stulpelinė diagrama}}
    \end{center}
\end{figure}

Remiantis diagrama didžiausias Tbx5 transkripcijos faktoriaus regionų skaičius
nustatytas eksperimento \small\emph{mm\_4\_emb\_fibr\_r1} techninėje replikoje,
kurioje pelių embrionų fibroblastų ląstelės dvi dienas veiktos AGHMT
faktoriais. Šį rezultatą palyginus su kitomis biologinėmis replikomis, kuriose
tirtas tas pats pelių embrionų fibroblastų ląstelių kamienas, tačiau ląstelės
veiktos tik kai kuriais faktoriais, pastebimas gradualus Tbx5 transkripcijos
faktoriaus regionų skaičiaus mažėjimas diagramoje
\small\emph{mm\_4\_emb\_fibr\_r2}, \small\emph{mm\_4\_emb\_fibr\_r3} ir
\small\emph{mm\_4\_emb\_fibr\_r4} pavaizduotuose stulpeliuose.

Mėginyje, kuriame embrionų fibroblastai veikti tik vienu faktoriumi, Tbx5
transkripcijos faktoriaus regionų nustatyta nedaug - 13520.

Mažiausiai regionų nustatyta mėginyje, kuriame tirta pelių naujagimių širdies
fibroblastų, ekspresuojančių T antigeną ir paveiktų inhibitoriais: sb431542 ir
xav939. Nepaisant to, kad abu inhibitoriai skatina širdies ląstelių
diferenciaciją\cite{HEART_CELL_DIFF_ARTCL}, itin mažas transkripcijos
faktoriaus regionų skaičius rodo, kad papildomas veikimas inhibitoriais daro
mažą įtaką transkripcijos faktoriaus jungimuisi prie DNR sekų.

\newpage

%%%%%%%%%%%%%%%%%%%%%%%%%%%%%%%%%%%%%%%%%%%%%
% REGIONŲ SKAIČIUS ATSKIROSE CHROMOSOMOSE
%%%%%%%%%%%%%%%%%%%%%%%%%%%%%%%%%%%%%%%%%%%%%

\subsection{Regionų pasiskirstymas chromosomose}
Nustačius Tbx5 transkripcijos faktoriaus regionų pasiskirstymą eksperimentų
mėginiuose, kitame analizės etape patikrinta, kaip faktoriaus regionai
pasiskirstę atskirose chromosomose.

Vaizduojamuose grafikuose (3 pav.) didžiausias regionų skaičius nustatomas
pirmoje, antroje ir penktoje chromosomose. Naminės pelės pirmoji chromosoma yra
pati didžiausia, turinti 195 milijonų bazių porų, antroji chromosoma sudaryta
iš 182 megabazių, penktoji chromosoma - 152 milijonų bazių porų, todėl didesnis
regionų skaičius šiose chromosomose nėra neįprastas reiškinys. Kitose
chromosomose regionų skaičius yra mažesnis. Ypač mažas regionų skaičius
nustatytas devynioliktoje (61 Mbp), X (169 Mbp) ir Y (91 Mbp) chromosomose.

\begin{figure}[htb]
    \begin{center}
        \includegraphics[width=0.8\linewidth]{../Figures/peak_counts_by_chr.png}
        \caption*{\small\textbf{3 pav. Regionų pasiskirstymas chromosomose}}
    \end{center}
\end{figure}

Biologinių replikų mėginiuose didžiausias regionų skaičius nustatytas antroje
chromosomoje. Taip pat grafikuose išsiskiria kontrolinis HL - 1 širdies
ląstelių mėginys, kuriame didžiausias regionų skaičius nustatytas penktojoje
chromosomoje.

Remiantis pavaizduotomis regionų skaičiaus pasiskirstymo chromosomose
stulpelinėmis diagramomis, itin išsiskiriantis atrankumas chromosomų atžvilgiu
nenustatytas, todėl galima teigti, jog šiame analizės etape duomenų
problematiškumas nepastebimas arba jo nėra.

\newpage

%%%%%%%%%%%%%%%%%%%%%%%%%%%%%%%%%%%%%%%%%%%
% TARP MĖGINIŲ PERSIDENGIANTYS REGIONAI
%%%%%%%%%%%%%%%%%%%%%%%%%%%%%%%%%%%%%%%%%%%

\subsection{Tarp mėginių persidengiantys regionai}
Dažnai siekiant nustatyti mėginių panašumą, yra tiriama, kokia mėginių duomenų
dalis persidengia.

\begin{figure}[htb]
    \begin{center}
        \includegraphics[width=0.7\linewidth]{../Figures/peak_overlaps.png}
        \caption*{\small\textbf{4 pav. Persidengiančių regionų procentinės
                                dalies spalvų intensyvumo grafikas}}
    \end{center}
\end{figure}

Remiantis pavaizduoto spalvų intensyvumo grafiko (4 pav.) duomenimis, didžiausi
persidengiančių regionų procentai nustatyti tarp šių mėginių:

\begin{itemize}
    \item \textbf{81.577 \%} - tarp mėginio, kuriame buvo tiriamos širdies
        fibroblastų ląstelės, ekspresuojančios T antigeną, ir mėginio, kuriame
        tirti embrionų fibroblastai, veikiant AGHMT.
    \item \textbf{81.092 \%} - tarp mėginio, kuriame tirti embrionų
        fibroblastai ir mėginio, kuriame nebuvo AKT1.
    \item \textbf{77.827 \%} - tarp mėginio su T antigeną ekspresuojančiomis
        širdies fibroblastų ląstelėmis ir mėginio, kuriame nebuvo AKT1.
    \item \textbf{76.948 \%} - tarp mėginio, kuriame nebuvo HAND2 faktoriaus,
        ir mėginio, kuriame nebuvo AKT1.
    \item \textbf{75.439 \%} - tarp mėginio su T antigeną ekspresuojančiomis
        širdies fibroblastų ląstelėmis ir tarp mėginio, kuriame nebuvo HAND2
        faktoriaus.
  \end{itemize}

%%%%%%%%%%%%%%%%%%%%%%%%%%%
% TBX5 MOTYVO NUSTATYMAS
%%%%%%%%%%%%%%%%%%%%%%%%%%%

\subsection{Tbx5 motyvo pasiskirstymas mėginiuose}
Penktajame grafike (5 pav.) pavaizduota, kiek Tbx5 motyvo atitikimų nustatyta
skirtinguose mėginiuose.

\begin{figure}[htb]
    \begin{center}
        \includegraphics[width=0.6\linewidth]{../Figures/tf_hit_percentage.png}
        \caption*{\small\textbf{5 pav. Tbx5 motyvų atitikimų skaičiaus palyginimo
                                sudėtinė diagrama}}
    \end{center}
\end{figure}

Daugiausiai Tbx5 motyvo sekos (AGGTGTCA) atitikimų (39820) nustatyta mėginyje,
kuriame embrionų fibroblastai veikti serino/treonino kinaze 1 (Akt1) bei
keliais transkripcijos faktoriais (GATA4, HAND2, MEF2C, Tbx5).

Mažiausias Tbx5 motyvo sekų skaičius nustatytas mėginiuose, kuriuose embrionų
fibroblastai veikti tik vienu transkripcijos faktoriumi
(\small\emph{mm\_4\_emb\_fibr\_r4}), ir mėginyje, kuriame pelių naujagimių
širdies fibroblastai veikti sb431542 ir xav939 inhibitoriais
(\small\emph{mm\_3\_cardiac\_fibr\_r1}). Nepaisant to, kad šiuose mėginiuose
motyvų skaičius mažiausias, remiantis procentine motyvų mėginiuose dalimi,
mėginys (\small\emph{mm\_3\_cardiac\_fibr\_r1}) Tbx5 motyvo sekos fragmentų turi
daug (1610), atsižvelgus į bendrą šio mėginio sekų kiekį.

Mėginio, kuriame tirtos širdies raumens ląstelės
(\small\emph{mm\_2\_cardiac\_muscle\_r1}), Tbx5 motyvų sekų nustatyta mažai,
palyginus identifikuoto transkripcijos faktoriaus motyvo sekų skaičių su bendru
šio mėginio sekų rinkinio dydžiu.

\newpage

\subsection{\emph{De novo} identifikuoti motyvai}
\emph{De novo} motyvų paieškos programos įvykdymas buvo ilgiausiai trukęs
analizės etapas, lyginat su kitais  tyrimo žingsniais. Šio etapo metu buvo
sugeneruoti HTML formato failai, kuriuose buvo pateiktas identifikuotų motyvų
sąrašas, išrikiuotas pagal p-vertę didėjančia tvarka, motyvų sekų logotipai,
nuorodos į puslapius su pozicinėmis motyvų svorių matricomis bei identifikuotų
motyvų procentinę dalį visame mėginio sekų rinkinyje, pateiktame FASTA formatu.

Trečioje lentelėje (3 lentelė) kiekvienam mėginiui pavaizduoti trys motyvai,
turintys mažiausią p-vertę bei apimantys didžiausią mėginių pilno sekų rinkinio
dalį (procentiškai). Penktame trečios lentelės stulpelyje nurodyta, kokią
procentinę dalį mėginyje sudaro identifikuotas Tbx5 motyvas.

\begin{table}[htb]
    \newcolumntype{M}[1]{>{\centering\arraybackslash}m{#1}}
    \small
    \caption*{\small\textbf{3 lentelė. Identifikuotų motyvų pavyzdžiai}}
    \begin{tabular}{|c|c|c|c|c|}
    \hline
    \textbf{Mėginys} & \textbf{Pavadinimas} & \textbf{p vertė} &
                       \textbf{\thead{Procentinė\\ dalis}} &
                       \textbf{\emph{Tbx5} motyvas} \\
    \hlineB{2.5}
    \multirow{3}{*}{\textbf{mm\_2\_cardiac\_muscle\_r1}} & Tbx6(T-box) &
                    1e-1881 & 20.28\% &
                    \multirow{3}{*}{\thead{1e-3266;\\ 54.87\%}} \\
    \cline{2-4}            & Tbet(T-box) & 1e-1472 & 16.48\% & \\
    \cline{2-4}            & Eomes(T-box) & 1e-1332 & 25.49\% & \\
    \hlineB{2.5}
    \multirow{3}{*}{\textbf{mm\_4\_emb\_fibr\_r1}} & Mef2b(MADS) &
                    1e-3037 & 15.60\% &
                    \multirow{3}{*}{\thead{1e-2359;\\ 44.22\%}} \\
    \cline{2-4}              & TRPS1(Zf) & 1e-2983 & 31.99\% & \\
    \cline{2-4}              & GATA3(Zf) & 1e-2936 & 25.49\% & \\
    \hlineB{2.5}
    \multirow{3}{*}{\textbf{mm\_4\_emb\_fibr\_r2}} & Fos(bZIP) &
                    1e-2912 & 13.22\% &
                    \multirow{3}{*}{\thead{1e-1873;\\ 42.97\%}} \\
    \cline{2-4}             & Fra1(bZIP) & 1e-2889 & 12.66\% & \\
    \cline{2-4}             & Fra2(bZIP) & 1e-2855 & 11.36\% & \\
    \hlineB{2.5}
    \multirow{3}{*}{\textbf{mm\_4\_emb\_fibr\_r3}} & GATA3(Zf) &
                    1e-1895 & 25.06\% &
                    \multirow{3}{*}{\thead{1e-1391;\\ 41.04\%}} \\
    \cline{2-4}              & TRPS1(Zf) & 1e-1872 & 31.55\% & \\
    \cline{2-4}              & Fos(bZIP) & 1e-1857 & 11.66\% & \\
    \hlineB{2.5}
    \multirow{3}{*}{\textbf{mm\_1\_heart\_r1}} & Mef2c(MADS) &
                    1e-1226 & 8.45\% &
                    \multirow{3}{*}{\thead{1e-438;\\ 39.08\%}} \\
    \cline{2-4}            & Mef2b(MADS) & 1e-1174 & 12.63\% & \\
    \cline{2-4}            & Mef2d(MADS) & 1e-1174 & 5.42\% & \\
    \hlineB{2.5}
    \multirow{3}{*}{\textbf{mm\_4\_emb\_fibr\_r4}} & TRPS1(Zf) &
                    1e-1404 & 63.53\% &
                    \multirow{3}{*}{\thead{1e0;\\ 26.57\%}} \\
    \cline{2-4}              & GATA3(Zf) & 1e-1271 & 52.43\% & \\
    \cline{2-4}              & GATA4(Zf) & 1e-1024 & 38.94\% & \\
    \hlineB{2.5}
    \multirow{3}{*}{\textbf{mm\_3\_cardiac\_fibr\_r1}} & Tbx6(T-box) &
                    1e-1431 & 38.28\% &
                    \multirow{3}{*}{\thead{1e-1474;\\ 69.71\%}} \\
    \cline{2-4}            & Tbet(T-box) & 1e-1077 & 33.27\% & \\
    \cline{2-4}            & Tbx21(T-box) & 1e-1034 & 30.60\% & \\
    \hline
    \end{tabular}
\end{table}

Remiantis trečios lentelės duomenimis bei naudojantis statistinių hipotezių
testavimu itin mažos \emph{p} vertės rodo, jog tikimybė, kad identifikuoti
motyvai atsitiktiniai, yra labai maža, todėl gauti duomenys yra statistiškai
reikšmingi - juos galima toliau analizuoti ir daryti įvairias išvadas.

Pirmajame ir paskutiniame mėginiuose, kurie lentelėje pažymėti
\small\emph{mm\_2\_cardiac\_muscle\_r1} ir
\small\emph{mm\_3\_cardiac\_fibr\_r1}, Tbx5 motyvo \emph{p}
vertės buvo mažiausios, o procentinė dalis - didžiausia. Palyginus šių mėginių
rezultatus su 4.4 dalyje gautais tų pačių mėginių rezultatais pastebimas
didelis procentinės dalies neatitikimas, tačiau šie procentai apskaičiuoti,
naudojantis skirtingais metodais. Taip pat 4.4 dalyje naudota pozicinė svorių
matrica neatitinka šioje dalyje identifikuoto Tbx5 motyvo pozicinės svorių
matricos.

Pateiktoje stulpelinėje diagramoje (6 pav.) vaizduojamas bendras identifikuotų
motyvų skaičius kiekvienam mėginiui.

\begin{figure}[htb]
    \begin{center}
        \includegraphics[width=0.7\linewidth]{../Figures/motifs_in_samples.png}
        \caption*{\small\textbf{6 pav. Identifikuotų motyvų skaičiaus
                                mėginiuose stulpelinė diagrama}}
    \end{center}
\end{figure}

Nėra neįprasta, kad paskutinių mėginių (\small\emph{mm\_4\_emb\_fibr\_r4} ir
\small\emph{mm\_3\_cardiac\_fibr\_r1}) identifikuotų motyvų skaičius yra pats
mažiausias - šie mėginiai turi mažiausią regionų skaičių bei mažiausią juos
atitinkančių sekų rinkinį.

Nepaisant šio atitikimo, mėginys \small\emph{mm\_1\_heart\_r1}, turintis
mažiausią regionų skaičių po \small\emph{mm\_4\_emb\_fibr\_r4} ir
\small\emph{mm\_3\_cardiac\_fibr\_r1} mėginių, turi didžiausią identifikuotų
motyvų skaičių (345 motyvai). Nustatyti motyvai yra \(\sim\)12 nukleotidų ilgio.

Toliau pateikiamame grafike (7 pav.) vaizduojama, kiek vienodų motyvų buvo
nustatyta skirtinguose mėginiuose.

\newpage

\begin{figure}[htb]
    \begin{center}
        \includegraphics[width=1\linewidth]{../Figures/motif_intersections.png}
        \caption*{\small\textbf{7 pav. Tarp mėginių persidengiančių motyvų
                                kompleksinė diagrama}}
    \end{center}
\end{figure}

Remiantis gautu grafiku, galima pastebėti, kad tarp mėginių porų buvo
nustatytas didelis visiems mėginiams būdingų motyvų skaičius. Patikrinus, kiek
motyvų aptinkami visuose mėginiuose, buvo nustatyta, kad 131 motyvas yra
būdingas visiems analizuojamiems mėginiams.

\subsection{\emph{De novo} nustatytų motyvų biologinės funkcijos}
Išsiaiškinus, kiek motyvų pateiktų mėginių regionų failuose buvo identifikuota,
buvo nustatytos su šiais motyvais susijusių genų funkcijos.

Pateiktame funkcijų sąraše akronimas \textbf{MF} nurodo molekulinę funkciją,
\textbf{BP} - biologinį procesą, \textbf{CC} - ląstelės komponentą. Naudojantis
UniProtKB\cite{UNIPROTKB} duomenų bazės rezultatais, visi trečioje lentelėje
(3 lentelė) pateikti motyvai:

\begin{itemize}
    \item \textbf{MF:} atlieka jungimosi prie DNR funkciją bei dalyvauja
        transkripcijos procese.
    \item \textbf{BP:} dalyvauja ląstelių diferenciacijos procesuose.
    \item \textbf{CC:} yra aptinkami branduolyje.
\end{itemize}

\newpage

Unikalios motyvų funkcijos aprašytos sąraše:

\begin{enumerate}
    \item \textbf{Tbx6(T-box)\cite{TBX6} - T-box transkripcijos faktorius 6}
        \begin{itemize}
            \item \textbf{BP:} dalyvauja ląstelių proliferacijos ir
                organizacijos procesuose, signalinių kelių valdyme,
                kardioblastų diferenciacijoje.
            \item \textbf{CC:} nėra prisijungęs prie membranų, aptinkamas
                branduolyje.
        \end{itemize}

    \item \textbf{Tbet(T-box)\cite{TBET} - T-box transkripcijos faktorius 21}
        \begin{itemize}
            \item \textbf{BP:} dalyvauja imuninės sistemos, baltymų metabolinių
                kelių procesuose. Taip pat pasireiškia organizmui reaguojant į
                dirgiklius.
            \item \textbf{CC:} aptinkamas branduolyje, neuroninių ląstelių kūne.
        \end{itemize}

    \item \textbf{Eomes(T-box)\cite{EOMES} - Eomesoderminas}
        \begin{itemize}
            \item \textbf{BP:} dalyvauja įvairių ląstelių (pavyzdžiui,
                kardiomiocitų) diferenciacijoje, neurogenezėje, kamieninių
                ląstelių populiacijos palaikyme
            \item \textbf{CC:} aptinkamas branduolyje, chromatine.
        \end{itemize}

    \item \textbf{Mef2b(MADS)\cite{MEF2B} - Miocitams specifiškas
                  transkripcijos aktyvatorius 2B}
        \begin{itemize}
            \item \textbf{BP:} dalyvauja įvairių ląstelių diferenciacijoje,
                transkripcijos proceso aktyvavime.
            \item \textbf{CC:} aptinkamas citozolyje, branduolyje,
                nukleoplazmoje, ląstelių jungtyse.
        \end{itemize}

    \item \textbf{Mef2c(MADS)\cite{MEF2C} - Miocitams specifiškas
                  transkripcijos aktyvatorius 2C}
        \begin{itemize}
            \item \textbf{BP:} dalyvauja ląstelių apoptozėje, kraujagyslių
                formavimęsi, pradinės embrionų širdies vystymęsi.
            \item \textbf{CC:} aptinkamas citozolyje, branduolyje,
                nukleoplazmoje, sarkomerose, sarkoplazmoje.
        \end{itemize}

    \item \textbf{Mef2d(MADS)\cite{MEF2D} - Miocitams specifiškas
                  transkripcijos aktyvatorius 2D}
        \begin{itemize}
            \item \textbf{BP:} dalyvauja suaugusių organizmų širdies vystymęsi,
                kremzlinių bei kaulinių ląstelių diferenciacijoje.
            \item \textbf{CC:} aptinkamas citoplazmoje, branduolyje,
                nukleoplazmoje.
        \end{itemize}

    \item \textbf{TRPS1(Zf)\cite{TRPS1} - Cinko „pirštelio”
                  transkripcijos faktorius}
        \begin{itemize}
            \item \textbf{BP:} dalyvauja kremzlinių ląstelių diferenciacijoje,
                skeleto vystymęsi, būdingas neigiamas transkripcijos
                reguliavimas.
            \item \textbf{CC:} aptinkamas branduolyje, nukleoplazmoje,
                chromatine, baltymų kompleksuose.
        \end{itemize}

    \item \textbf{GATA3(Zf)\cite{GATA3} - T ląstelėms specifiškas
                  transkripcijos faktorius GATA-3}
        \begin{itemize}
            \item \textbf{BP:} dalyvauja aortos vožtuvų formavimęsi, širdies
                prieširdžių morfogenezėje, embrionų organų vystymęsi,
                eritrocitų diferenciacijoje.
            \item \textbf{CC:} aptinkamas branduolyje, nukleoplazmoje,
                chromatine.
        \end{itemize}

    \item \textbf{GATA4(Zf)\cite{GATA4} - Transkripcijos faktorius GATA-4}
        \begin{itemize}
            \item \textbf{BP:} dalyvauja širdies ląstelių diferenciacijoje,
                širdies raumens regeneracijoje, embrionų širdies formavimęsi.
            \item \textbf{CC:} aptinkamas branduolyje, nukleoplazmoje,
                chromatine.
        \end{itemize}

    \item \textbf{Fos(bZIP)\cite{FOS} - AP-1 transkripcijos
                  faktoriaus subvienetas}
        \begin{itemize}
            \item \textbf{BP:} dalyvauja atsako į jonus (kadmio, kalcio),
                citokinus bei progesteroną procesuose. Taip pat aktyvus nervų
                sistemos vystymosi metu.
            \item \textbf{CC:} aptinkamas citozolyje, branduolyje,
                nukleoplazmoje, endoplazminiame tinkle, sinaptosomose.
        \end{itemize}

    \item \textbf{Fra1(bZIP)\cite{FRA1} - Onkogenas, AP-1 transkripcijos
                  faktoriaus subvienetas}
        \begin{itemize}
            \item \textbf{BP:} dalyvauja ląstelės ciklo valdyme, apoptoziniuose
                procesuose bei embrionų vystymosi gimdoje procesuose.
            \item \textbf{CC:} aptinkamas citozolyje, branduolyje,
                nukleoplazmoje, presinapsinėje membranoje.
        \end{itemize}

    \item \textbf{Fra2(bZIP)\cite{FRA2} - AP-1 transkripcijos
                  faktoriaus subvienetas}
        \begin{itemize}
            \item \textbf{BP:} dalyvauja teigiamoje fibroblastų
                proliferacijoje, atsako į estradiolio - moteriško
                lytinio hormono - procesuose. Taip pat būdinga teigiamas
                transkripcijos reguliavimas.
            \item \textbf{CC:} aptinkamas branduolyje ir nukleoplazmoje.
        \end{itemize}
\end{enumerate}

\newpage

%%%%%%%%%%%%
% IŠVADOS
%%%%%%%%%%%%

\section{Išvados}
Atlikus Tbx5 transkripcijos faktoriaus analizę su skirtingomis naminės pelės
ląstelėmis, kurioms buvo taikyti skirtintingi poveikiai, gauti rezultatai buvo
apibendrinti:

\begin{itemize}
    \item Didžiausias praturtintų genominių regionų skaičius nustatytas naminės
        pelės embrionų fibroblastų ląstelėse (163190 regionų), kurios veiktos
        AGHMT. Mažiausias praturtintų genominių regionų skaičius nustatytas
        pelių naujagimių širdies fibroblastų, veiktų sb431542 ir xav939
        inhibitoriais, linijoje (13440 regionai).
    \item Skirtingose genominėse pozicijose (chromosomose) Tbx5 faktoriaus
        regionų pasiskirstymas yra susijęs su naminių pelių chromosomų dydžiais.
        Didžiausias Tbx5 regionų skaičius nustatytas pirmoje ir antroje
        chromosomose, kurios yra sudarytos iš daugiausiai nukleotidų.
        Mažiausias regionų skaičius nustatytas lytinėse chromosomose - X ir Y,
        kur Y chromosomoje Tbx5 regionų skaičius neviršijo 22 regionų skaičiaus.
    \item Didžiausias mėginių persidengimo procentas nustytatas tarp mėginio,
        kuriame tirti širdies fibroblastai veikti inhibitoriais ir AGHMT
        veiktų fibroblastų mėginio (81.577\%). Taip pat didelis panašumas
        (81.092\%) nustatytas tarp mėginių, kuriuose naminių pelių embrionai
        veikti pilnu serino/treonino kinazės 1 (Akt1) ir transkripcijos
        faktorių GATA4, HAND2, MEF2C, TBX5 rinkiniu bei rinkiniu, kuriame
        nebuvo Akt1, tačiau buvo transkripcijos faktorių.
\end{itemize}


\newpage

%%%%%%%%%%%%%%%%%%%%%%%%%%
% LITERATŪROS ŠALTINIAI
%%%%%%%%%%%%%%%%%%%%%%%%%%

\bibliographystyle{plain}
\begin{thebibliography}{99}

\bibitem{REGENERATION} Kate MacCord, Jane Maienschein (2019) Philosophy of
Biology: Understanding regeneration at different scales eLife 8:e46569
https://doi.org/10.7554/eLife.46569

\bibitem{ORGANISMS} Mehta AS, Singh A. Insights into regeneration tool box:
An animal model approach. Dev Biol. 2019 Sep 15;453(2):111-129.
doi: 10.1016/j.ydbio.2019.04.006. Epub 2019 Apr 13. PMID: 30986388;
PMCID: PMC6684456.

\bibitem{GTRD} GTRD: an integrated view of transcription regulation.
Kolmykov S, Yevshin I, Kulyashov M, Sharipov R, Kondrakhin Y, Makeev VJ,
Kulakovskiy IV, Kel A, Kolpakov F Nucleic Acids Res. 2021 Jan
8;49(D1):D104-D111.

\bibitem{UCSCGB} UCSC Genome Browser: Kent WJ, Sugnet CW, Furey TS, Roskin KM,
Pringle TH, Zahler AM, Haussler D. The human genome browser at UCSC. Genome Res.
2002 Jun;12(6):996-1006.

\bibitem{ARTCL1} He A, Kong SW, Ma Q, Pu WT. Co-occupancy by multiple cardiac
transcription factors identifies transcriptional enhancers active in heart.
Proc Natl Acad Sci U S A. 2011 Apr 5;108(14):5632-7.
doi: 10.1073/pnas.1016959108. Epub 2011 Mar 17. PMID: 21415370;
PMCID: PMC3078411.

\bibitem{ARTCL2} Hashimoto H, Wang Z, Garry GA, Malladi VS, Botten GA, Ye W,
Zhou H, Osterwalder M, Dickel DE, Visel A, Liu N, Bassel-Duby R, Olson EN.
Cardiac Reprogramming Factors Synergistically Activate Genome-wide Cardiogenic
Stage-Specific Enhancers. Cell Stem Cell. 2019 Jul 3;25(1):69-86.e5.
doi: 10.1016/j.stem.2019.03.022. Epub 2019 May 9. PMID: 31080136;
PMCID: PMC6754266.

\bibitem{ARTCL3} Stone NR, Gifford CA, Thomas R, Pratt KJB, Samse-Knapp K,
Mohamed TMA, Radzinsky EM, Schricker A, Ye L, Yu P, van Bemmel JG, Ivey KN,
Pollard KS, Srivastava D. Context-Specific Transcription Factor Functions
Regulate Epigenomic and Transcriptional Dynamics during Cardiac Reprogramming.
Cell Stem Cell. 2019 Jul 3;25(1):87-102.e9. doi: 10.1016/j.stem.2019.06.012.
PMID: 31271750; PMCID: PMC6632093.

\bibitem{JCKSLAB} Jackson Laboratory (RRID:SCR\_004633).

\bibitem{R} R Core Team (2022).
R: A language and environment for statistical computing. R Foundation
for Statistical Computing, Vienna, Austria. URL https://www.R-project.org/.

\bibitem{SCIK}Scikick. Utility for executing collections of computational
notebooks.\\
URL https://petronislab.camh.ca/pub/scikick/stable/docs/report/out\_html/introduction.html

\bibitem{R_TRACK} M. Lawrence, R. Gentleman, V. Carey: "rtracklayer: an {R}
package for interfacing with genome browsers". Bioinformatics 25:1841-1842.

\bibitem{R_GGPLOT} H. Wickham. ggplot2: Elegant Graphics for Data Analysis.
Springer-Verlag New York, 2016.

\bibitem{BBTOBED} BigWig and BigBed tools: Kent WJ, Zweig AS, Barber G,
Hinrichs AS, Karolchik D. BigWig and BigBed: enabling browsing of large
distributed data sets. Bioinformatics. 2010 Sep 1;26(17):2204-7.

\bibitem{BEDTOOLS} Quinlan AR, Hall IM. BEDTools: a flexible suite of
utilities for comparing genomic features. Bioinformatics. 2010 Mar
15;26(6):841-2. doi: 10.1093/bioinformatics/btq033. Epub 2010 Jan 28.
PMID: 20110278; PMCID: PMC2832824.

\bibitem{GET_FASTA} BEDTools komandinės eilutės įrankis. Programų rinkinio
programa \emph{getfasta}.\\
Prieiga per https://bedtools.readthedocs.io/en/latest/content/tools/getfasta.html
[žiūrėta 2022-06-03].

\bibitem{BIOSTR} Pagès H, Aboyoun P, Gentleman R, DebRoy S (2022). \_Biostrings:
Efficient manipulation of biological strings\_. R package version
2.64.0, <https://bioconductor.org/packages/Biostrings>.

\bibitem{HOCOMOCO} HOCOMOCO: towards a complete collection of transcription
factor binding models for human and mouse via large-scale ChIP-Seq analysis
Ivan V. Kulakovskiy; Ilya E. Vorontsov; Ivan S. Yevshin; Ruslan N. Sharipov;
Alla D. Fedorova; Eugene I. Rumynskiy; Yulia A. Medvedeva; Arturo Magana-Mora;
Vladimir B. Bajic; Dmitry A. Papatsenko; Fedor A. Kolpakov; Vsevolod J. Makeev
Nucl. Acids Res., Database issue, gkx1106 (11 November 2017)
doi: 10.1093/nar/gkx1106

\bibitem{HOMER} Heinz S, Benner C, Spann N, Bertolino E et al. Simple
Combinations of Lineage-Determining Transcription Factors Prime cis-Regulatory
Elements Required for Macrophage and B Cell Identities. Mol Cell 2010 May
28;38(4):576-589. PMID: 20513432.

\bibitem{UPSETR} Jake R. Conway, Alexander Lex, Nils Gehlenborg. UpSetR: An R
Package For The Visualization Of Intersecting Sets And Their Properties
Bioinformatics, 33(18): 2938-2940, doi:10.1093/bioinformatics/btx364, 2017.

\bibitem{UNIPROT} The UniProt Consortium UniProt: the universal protein
knowledgebase in 2021 Nucleic Acids Res. 49:D1 (2021).

\bibitem{UNIPROTKB} Boutet E, Lieberherr D, Tognolli M, Schneider M, Bairoch A.
UniProtKB/Swiss-Prot Methods Mol. Biol. 406:89-112 (2007)

\bibitem{HEART_CELL_DIFF_ARTCL} Drowley L, Koonce C, Peel S, et al. Human
Induced Pluripotent Stem Cell-Derived Cardiac Progenitor Cells in Phenotypic
Screening: A Transforming Growth Factor-β Type 1 Receptor Kinase Inhibitor
Induces Efficient Cardiac Differentiation. Stem Cells Transl Med.
2016;5(2):164-174. doi:10.5966/sctm.2015-0114

\bibitem{TBX6} UniProtKB duomenų bazė. \emph{Tbx6 aprašymas} (2022).\\
Prieiga per https://www.uniprot.org/uniprot/P70327 [žiūrėta 2022-06-12].

\bibitem{TBET} UniProtKB duomenų bazė. \emph{Tbet aprašymas} (2022).\\
Prieiga per https://www.uniprot.org/uniprot/Q9JKD8 [žiūrėta 2022-06-12].

\bibitem{EOMES} UniProtKB duomenų bazė. \emph{Eomes aprašymas} (2022).\\
Prieiga per https://www.uniprot.org/uniprot/O54839 [žiūrėta 2022-06-12].

\bibitem{MEF2B} UniProtKB duomenų bazė. \emph{Mef2b aprašymas} (2022).\\
Prieiga per https://www.uniprot.org/uniprot/O55087 [žiūrėta 2022-06-12].

\bibitem{MEF2C} UniProtKB duomenų bazė. \emph{Mef2c aprašymas} (2022).\\
Prieiga per https://www.uniprot.org/uniprot/Q8CFN5 [žiūrėta 2022-06-12].

\bibitem{MEF2D} UniProtKB duomenų bazė. \emph{Mef2d aprašymas} (2022).\\
Prieiga per https://www.uniprot.org/uniprot/Q63943 [žiūrėta 2022-06-12].

\bibitem{TRPS1} UniProtKB duomenų bazė. \emph{TRPS1 aprašymas} (2022).\\
Prieiga per https://www.uniprot.org/uniprot/Q925H1 [žiūrėta 2022-06-12].

\bibitem{GATA3} UniProtKB duomenų bazė. \emph{GATA3 aprašymas} (2022).\\
Prieiga per https://www.uniprot.org/uniprot/P23772 [žiūrėta 2022-06-12].

\bibitem{GATA4} UniProtKB duomenų bazė. \emph{GATA4 aprašymas} (2022).\\
Prieiga per https://www.uniprot.org/uniprot/Q08369 [žiūrėta 2022-06-12].

\bibitem{FOS} UniProtKB duomenų bazė. \emph{Fos aprašymas} (2022).\\
Prieiga per https://www.uniprot.org/uniprot/P01101 [žiūrėta 2022-06-12].

\bibitem{FRA1} UniProtKB duomenų bazė. \emph{Fra1 aprašymas} (2022).\\
Prieiga per https://www.uniprot.org/uniprot/P48755 [žiūrėta 2022-06-12].

\bibitem{FRA2} UniProtKB duomenų bazė. \emph{Fra2 aprašymas} (2022).\\
Prieiga per https://www.uniprot.org/uniprot/P47930 [žiūrėta 2022-06-12].

\end{thebibliography}

\newpage

%%%%%%%%%%%%
% PRIEDAI
%%%%%%%%%%%%

\section{Priedas}

Priedų sąraše pateikiamos tarpinių rezultatų puslapio, sugeneruoto su Scikick,
bei Git repozitorijos, kurioje saugomi analizei naudoti duomenų failai,
parašyti skriptai bei pagrindinė R programa, nuorodos.

\begin{itemize}
    \item \textbf{Tarpinių rezultatų Scikick puslapis:}\\
        https://karklas.mif.vu.lt/\(\sim\)dast6577/KursinisDarbas/v1.1/peaks\_MM.html
    \item \textbf{Analizės Git repozitorija:}\\
        https://github.com/dansta0804/TF\_analysis.git
  \end{itemize}

\end{document}
