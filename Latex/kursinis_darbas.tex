\documentclass[12pt]{article}
\usepackage{indentfirst}
\usepackage[utf8x]{inputenc}
\usepackage[T1]{fontenc}
\usepackage[english,lithuanian]{babel}
\usepackage{array}
\usepackage{caption}
\usepackage{makecell}
\usepackage[euler]{textgreek}
\usepackage{multirow}
\usepackage{boldline}
\usepackage{floatrow}
\floatsetup[table]{capposition=top}
\usepackage{amsmath, amsthm, amssymb}
\usepackage{graphicx}
\usepackage{setspace}
\usepackage{verbatim}
\usepackage[left=3cm,top=2cm,right=1.5cm,bottom=2cm]{geometry}
\usepackage{floatrow}
\newfloatcommand{capbtabbox}{table}[][\FBwidth]
\usepackage{blindtext}
\onehalfspacing
\usepackage[hidelinks]{hyperref}

\newcommand{\EE}{\mathbb{E}\,} % Mean
\newcommand{\ee}{{\mathrm e}}  % nice exponent
\newcommand{\dd}{{\mathrm d}}
\newcommand{\RR}{\mathbb{R}}

\begin{document}
\selectlanguage{lithuanian}

\begin{titlepage}
\vskip 20pt
\begin{center}
\includegraphics[scale=0.5]{MIF}
\end{center}

%%%%%%%%%%%%%%%%%%%%%%%%%%%%%%%
% TITULINIO PUSLAPIO TEKSTAS
%%%%%%%%%%%%%%%%%%%%%%%%%%%%%%%

\vskip 20pt
\centerline{\bf \large \textbf{VILNIAUS UNIVERSITETAS}}
\bigskip
\centerline{\large \textbf{MATEMATIKOS IR INFORMATIKOS FAKULTETAS}}
\bigskip
\centerline{\large \textbf{BIOINFORMATIKOS BAKALAURO STUDIJŲ PROGRAMA}}

\vskip 90pt
\begin{center}
    {\bf \LARGE Tbx5 transkripcijos faktoriaus tyrimas \emph{Mus musculus}
     širdies ląstelėse}
\end{center}
\begin{center}
    {\bf \Large Research of Tbx5 transcription factor in \emph{Mus musculus}
     heart cells}
\end{center}
\vskip 20pt
\centerline{\bf \large \textbf{Kursinis darbas}}
\bigskip
\vskip 40pt

\hskip 140pt {\large Autorė: Danielė Stasiūnaitė}

\hskip 140pt{\large VU el. p.: (daniele.stasiunaite@mif.stud.vu.lt)}
\bigskip
\vskip 20pt

\hskip 140pt {\large Darbo vadovė: J. m. d. Kotryna Kvederavičiūtė}
\vskip 60pt
\vskip 40pt
\centerline{\large \textbf{Vilnius}}
\centerline{\large \textbf{2022}}
\newpage
\end{titlepage}

\selectlanguage{lithuanian}

%%%%%%%%%%%%%%%%%%%%%
% TURINIO PUSLAPIS
%%%%%%%%%%%%%%%%%%%%% 
\tableofcontents
\newpage

%%%%%%%%%%%%%%%%%%%%%%%%%%%%%%%%%%%%
% LIETUVIŠKOS SANTRAUKOS PUSLAPIS
%%%%%%%%%%%%%%%%%%%%%%%%%%%%%%%%%%%%

\section*{Santrauka}
Širdies regeneracijos procesų tyrinėjimas yra svarbi sritis, galinti
prisidėti prie širdies ligų gydymo bei širdies audinių atstatymo po
pažeidimų (pvz., širdies raumens plyšimo). Sėkmingam ląstelių regeneracijos
taikymui praktikoje svarbu suprasti šio proceso mechanizmus bei išsiaiškinti
juose dalyvaujančius genus bei jų svarbą.

Tbx5 - \emph{Tbx5} geno koduojamas vienas iš daugelio kardiogeninių
transkripcijos faktorių, aktyvuojančių genus, kurie dalyvauja širdies
vystymosi embriogenezės metu bei audinio atstatyme regeneracijos metu.
                                                                                     
Šiame darbe, naudojantis R programavimo kalbos bibliotekomis bei
bioinformatiniais komandinės eilutės bei internetiniais duomenų apdorojimo
bei analizės įrankiais, išanalizuota, kokiuose ChIP sekoskaitos metodu gautų
naminės pelės (lot. \emph{Mus musculus}) ląstelių pikų mėginiuose Tbx5
transkripcijos faktoriaus motyvų yra daugiausiai bei kokie motyvai gali būti
identifikuoti analizuojamuose mėginiuose.

Atlikta statistinė duomenų analizė parodė, kad panašiose ląstelėse,
kurioms buvo taikytos skirtingos eksperimentinės sąlygos, ChIP sekoskaitos
duomenys bei Tbx5 transkripcijos faktoriaus motyvų skaičius skiriasi itin
stipriai, tačiau \emph{de novo} identifikuotuose motyvuose vyrauja bendra
tendencija - identifikuoti motyvai yra susiję su ląstelių diferenciacija
bei gali būti reikšmingi regeneracijos procese.

\hfill \break
\textbf{Raktiniai žodžiai:} regeneracija, ChIP sekoskaita (ChIP-seq),
    transkripcijos faktorius, Tbx5, motyvas, R.
\newpage

%%%%%%%%%%%%%%%%%%%%%%%%%%%%%%%%%%%%%%%%
% ANGLIŠKOS SANTRAUKOS PUSLAPIS
%%%%%%%%%%%%%%%%%%%%%%%%%%%%%%%%%%%%%%%%

\section*{Summary}
The investigation of regeneration processes is an important field of research
that may play a significant role in heart disease treatment and heart tissue
repair after physical heart damage (for example, cardiac rupture).
Therefore, in order to successfully apply cell regeneration methods in practice,
it is mandatory to determine and apprehend the mechanisms and gene
products that regulate regeneration processes.

Tbx5 - cardiogenic transcription factor that is encoded by \emph{Tbx5} gene.
This factor activates genes that are involved in heart development during
embryogenesis and tissue repair during regeneration.

In this work, the analysis of the ChIP sequencing samples that were
retrieved from different house mouse (lat. \emph{Mus musculus}) cells was
conducted using functions from R programming language libraries and performing
analysis' steps using bioinformatics' command-line tools and online services
in order to determine what type of house mouse cells has the greatest number
of Tbx5 transcription factor motif sequences and what other motifs can be
identified in the samples of interest.

The undertaken statistical analysis indicated that ChIP sequencing data and
Tbx5 transcription factor motif count notably differ between analyzed
samples. On the other hand, after performing \emph{de novo} motif discovery, it
was determined that identified motifs are all related to cell differentiation
and may contribute to regeneration process mechanisms.

\hfill \break
\textbf{Keywords:} regeneration, ChIP sequencing (ChIP-seq),
    transcription factor, Tbx5, motif, R.
\newpage

%%%%%%%%%%%%%%%%%%%
% ĮVADO PUSLAPIS
%%%%%%%%%%%%%%%%%%%

\section{Įvadas}
\subsection*{Darbo temos aktualumas}

Spartėjanti mokslo raida, įvairūs atradimai bei išradimai stipriai paspartino
ir pagerino įvairių ligų diagnostikos bei prevencijos tyrimus, praplėtė žinias
ląstelės biologijos, genetikos, fiziologijos ir kitose srityse. Viena iš
medicinos sričių, kuri pradėta tyrinėti dar XVIII a., kai buvo nustatyta, kad
kai kurie organizmai geba atsiauginti prarastas arba sužeistas galūnes bei
kitus kūno audinius\cite{REGENERATION}, yra organizmų audinių bei organų
regeneracija.

Šiais laikais regeneracijos tyrimai atliekami su hidromis, planarijomis,
tritonais bei zebražuvėmis\cite{ORGANISMS}, siekiant išsiaiškinti šių organizmų
audinių regeneracijos mechanizmus bei pritaikyti žmonėms, patyrusiems traumas
ar turintiems specifinių kūno audinių pažeidimų. Regeneracijos procese
pagrindinę funkciją atlieka kamieninės ląstelės bei įvairūs transkripcijos
reguliavimo faktoriai, gebantys prisijungti prie DNR chromatino ir skatinti
arba slopinti specifinių genų transkripciją.

Regeneracijos procesų supratimui ir taikymui būtina išsiaiškinti, kokie
transkripcijos faktoriai dalyvauja šiame procese bei kokias funkcijas jie
atlieka, todėl šio darbo metu atlikta kardiogeninio transkripcijos
faktoriaus - Tbx5 - analizė iš naminių pelių širdžių išgautų ląstelių.

\subsection*{Darbo tikslas}

Patikrinti, ar skirtingų poveikių taikymas \emph{Mus musculus} širdies
ląstelėms daro didelę įtaką Tbx5 transkripcijos faktoriaus motyvų skaičiui.

\subsection*{Uždaviniai}
\begin{itemize}
    \item Išanalizuoti straipsnius bei pasirinkti tinkamiausius tyrimo duomenis.
    \item Įvertinti pikų skirtumus tarp mėginių, taikant skirtingus statistinių
        duomenų vaizdavimo būdus.
    \item Nustatyti, kokioms širdies ląstelėms būdingas didžiausias Tbx5
        transkripcijos faktoriaus motyvo taikinių skaičius.
    \item Atlikus \emph{de novo} motyvų paiešką, nustatyti, kokios biologinės
        funkcijos yra būdingos identifikuotiems motyvams.
\end{itemize}

Atskiruose šio darbo poskyriuose pateikiami statistinius duomenų skirtumus
vizualizuojantys grafikai bei gauti Tbx5 ir \emph{de novo} motyvų
identifikavimo analizių rezultatai.

\newpage

%%%%%%%%%%%%%%%%%%%%%
% DUOMENŲ APŽVALGA
%%%%%%%%%%%%%%%%%%%%%

\section{Duomenų bazės ir duomenys}
\subsection{GTRD duomenų bazė}
Tyrimui naudoti duomenys atsisiųsti iš GTRD (Gene Transcription Regulation
Database)\cite{GTRD} duomenų bazės, saugančios informaciją apie transkripcijos
sekų ir atviro chromatino regionus. Taip pat duomenų bazėje saugomi
nekartografuojamų regionų duomenys bei potencialūs žmonių bei naminių pelių
regionai, prie kurių gali jungtis transkripcijos faktoriai.

Ši duomenų bazė pasirinkta dėl sistemiškai surinktų ChIP sekoskaitos
eksperimentų, kurių metu gauti rezultatai yra unifikuotai apdoroti ir
paruošti tyrėjų meta-analizėms.

GTRD duomenų bazėje duomenys saugomi binariniu anotacijų formatu bigBed,
leidžiančiu atvaizduoti pasirinktą chromosomos regioną interaktyviose genominės
informacijos vizualizavimo naršyklėse (pavyzdžiui, UCSC Genome
Browser\cite{UCSCGB}) efektyviau nei tekstinis BED formatas.

\subsection{Pasirinktų mėginių charakteristika}
Analizė atlikta, naudojantis 3 nepriklausomais eksperimentais, kuriuos iš viso
sudarė 7 skirtingi mėginiai. Pirmoje lentelėje (1 lentelė) pateikta informacija
apie tyrimui atlikti naudotus duomenis, surinktus iš naminių pelių
(lot. \emph{Mus musculus}) širdžių ląstelių.

\begin{table}[htb]
    \newcolumntype{M}[1]{>{\centering\arraybackslash}m{#1}}
    \small
    \caption*{\small\textbf{1 lentelė. Mėginių charakteristikos}}
    \begin{tabular}{|c|c|c|c|c|c|c|}
    \hline
    %\thead{Sample\\ window}
    \textbf{GTRD ID} & \textbf{Ląstelių tipas} &
        \textbf{\thead{Kamienas}} & \textbf{\thead{Poveikis}} &
        \textbf{Antikūnai} & \textbf{PubMed ID}\\
    \hline
    EXP030898 & \thead{HL - 1\\ (širdies raumens)} & C57BL/6J &
                \thead{TRE\\ promotorius (2 d.)} & - & 21415370\cite{ARTCL1}\\ 
    \hline
    EXP058852 & Širdies prieširdžių & C57BL/6 & - &
                \thead{Tbx5\\ (sc-17866)} & 31080136\cite{ARTCL2}\\
    \hline
    EXP062056 & \thead{Pelių naujagimių širdies\\ fibroblastų, 
                ekspresuojančių\\ didelį kiekį T antigeno, linija} & CD1 &
                \thead{sb431542,\\ xav939} & \thead{anti-TBX5\\ (sc-17866x)} &
                31271750\cite{ARTCL3}\\
    \hline
    EXP058843 & \thead{MEF\\ (embrionų fibroblastai)} & C57BL/6 &
                AGHMT (2 d.) & \thead{anti-Tbx5\\ (sc-17866)} &
                31080136\cite{ARTCL2}\\
    \hline
    EXP058847 & \thead{MEF\\ (embrionų fibroblastai)} & C57BL/6 & GHMT (2 d.) &
                \thead{Tbx5\\ (sc-17866)} & 31080136\cite{ARTCL2}\\
    \hline
    EXP058850 & \thead{MEF\\ (embrionų fibroblastai)} & C57BL/6 & GMT (2 d.) &
                \thead{Tbx5\\ (sc-17866)} & 31080136\cite{ARTCL2}\\
    \hline
    EXP058856 & \thead{MEF\\ (embrionų fibroblastai)} & C57BL/6 &
                \thead{vienas\\ faktorius (2 d.)} & \thead{Tbx5\\ (sc-17866)} &
                31080136\cite{ARTCL2}\\
    \hline
    \end{tabular}
\end{table}

\newpage

\subsection{Santrumpų bei pavadinimų paaiškinimai}
\begin{itemize}
    \item \textbf{HL - 1}: pelių širdies raumens ląstelės, išgautos iš
        navikinių prieširdžių kardiomiocitų linijos. Šios ląstelės gali
        betarpiškai dalintis ir spontaniškai keisti savo formą, vykstant
        širdies raumens susitraukimo/atsipalaidavimo procesams.
    \item \textbf{MEF}: pelių embrionų fibroblastai (angl. \emph{Mouse
        Embryonic Fibroblast}). Šiai ląstelių linijai būdingas ląstelių
        gyvybingumo apribojimas, reiškiantis, jog šios ląstelės greitai
        pasensta ir miršta.
    \item \textbf{C57BL/6}: inbrydingo (angl. \emph{inbreeding}) būdu išvestų
        naminių pelių veislė. Šios veislės pelėms būdingas itin tamsus
        kailis, padidėjęs jautrumas garsams, kvapams, skausmui ir žemai
        temperatūrai. Ši veislė dažnai naudojama nutukimą ir imuninę sistemą
        tiriančiuose tyrimuose.
    \item \textbf{C57BL/6J}: prie naminių pelių veislės pavadinimo pridėtos
        raidės patikslina, kurioje laboratorijoje veislės išvestos. 'J' raidė
        nurodo, kad pelių veislė išvesta Meino valstijoje (JAV) įsikūrusioje
        Džeksono laboratorijoje\cite{JCKSLAB}.
    \item \textbf{CD1}: autbrydingo (angl. \emph{outbreeding}) būdu išvestų
        naminių pelių veislė. Šios veislės pelėms būdingas baltas kailis.
        Taip pat CD1 pelės dažnai naudojamos genetiniuose, toksikologiniuose,
        farmakologiniuose ir senėjimo tyrimuose.
    \item \textbf{TRE}: tetraciklino atsako elementas (angl. \emph{Tetracycline
        Response Element}). Tai yra 7 DNR sekos fragmentai,
        sudaryti iš 19 nukleotidų ir atskirti trumpesniais sekų fragmentais.
    \item \textbf{sb431542}: stipriai veikianti, selektyvi cheminė medžiaga;
        augimo faktoriaus inhibitorius.
    \item \textbf{xav939}: stipriai veikianti cheminė medžiaga; tankirazės
        inhibitorius. Tankirazė slopina specialaus baltymo, stabdančio
        telomerazės veiklą, jungimąsi prie telomerinių DNR sekų.
    \item \textbf{AGHMT}: Akt1 - serino/treonino kinazė 1; GATA4, HAND2, MEF2C,
        Tbx5 - kardiogeniniai transkripcijos faktoriai.
    \item \textbf{GHMT}: GATA4, HAND2, MEF2C, Tbx5 transkripcijos faktorių
        komplektas.
    \item \textbf{GMT}: GATA4, MEF2C, Tbx5 transkripcijos faktorių komplektas.
    \item \textbf{sc-17866x/sc-17866}: iš ožkų išskirti antikūnai,
        atpažįstantys žmonių, pelių ir žiurkių Tbx5 antigeną.
\end{itemize}

\subsection{Pasirinktų mėginių apžvalga}
\begin{itemize}
    \item \textbf{EXP030898}: vienas mėginys iš septyniolikos eksperimento
        metu tirtų mėginių. Eksperimente buvo siekiama patvirtinti arba
        atmesti hipotezę apie širdies stipriklių (angl. \emph{enhancer})
        identifikiavimą prie chromatino jungiantis keliems transkripcijos
        faktoriams.

        HL - 1 širdies raumens ląstelės buvo infekuotos su adenovirusu
        bei pridėtu TRE promotoriumi, skatinančiu Tbx5 transkripcijos
        faktoriaus geno raišką.
    \item \textbf{EXP058852}: prieširdžių ląstelės buvo du kartus po 24
        valandas laikytos mišinyje su retrovirusais. Papildomi poveikiai nebuvo
        taikyti.
    \item \textbf{EXP062056}: eksperimente ląstelės buvo infekuotos su GATA4,
        Mef2c, ir Tbx5 transkripcijos faktorius sintetinančiais retrovirusais.
        Ląstelės augintos terpėje, kurioje buvo inhibitoriaus sb431542,
        skatinančio kardiomiocitų diferenciaciją iš pliuripotentinių
        kamieninių ląstelių, ir inhibitoriaus xav939, stabdančio
        nediferencijuotų ląstelių sintezę ir skatinančio progenitorinių
        ląstelių kardiomiogenezę.
\end{itemize}

Pasirinktų duomenų rinkinyje naudoti vieno eksperimento, kuriame buvo tirti
širdies ląstelių atsinaujinimo ir diferenciacijos mechanizmai, keturiais
skirtingais poveikiais tirti embrionų fibroblastų mėginiai:

\begin{itemize}
    \item \textbf{EXP058843}: embrionų fibroblastai veikti AGHMT. Esant
        prisijungusiems kardiogeniniams transkripcijos faktoriams, Akt1
        skatina fibroblastų diferenciaciją į širdies ląsteles - kardiomiocitus.
    \item \textbf{EXP058847}: neįtraukta Akt1 serino/treonino kinazė 1.
    \item \textbf{EXP058850}: į kardiogeninių transkripcijos faktorių mišinį
        neįtrauktas HAND2 transkripcijos faktorius.
    \item \textbf{EXP058856}: ląstelės veiktos tik vienu kardiogeniniu 
        transkripcijos faktoriumi.
  \end{itemize}
\newpage

%%%%%%%%%%%%
% METODAI
%%%%%%%%%%%%

\section{Tyrimo metodai}
Tbx5 transkripcijos faktoriaus pikų tyrimo analizė atlikta su R programavimo
kalba\cite{R} (4.2.0 versija).

Tarpiniams analizės rezultatams pateikti naudotas komandinės eilutės įrankis
Scikick\cite{SCIK} (0.2.0 versija), leidžiantis generuoti R Markdown (Rmd)
ataskaitas HTML formatu bei kurti struktūrizuotus puslapius, apjungiant
iš daugelio Rmd failų gautus HTML ataskaitų failus.

\subsection{Pikų skaičiaus nustatymas mėginiuose}
Pikų skaičius skirtinguose mėginiuose apskaičiuotas su
standartine R ilgio funkcija \emph{length()}, kuri pritaikyta \emph{GRanges}
objektui, aprašančiam genomines pozicijas bei su jomis susijusias anotacijas.
Objektas sukurtas su \emph{rtracklayer}\cite{R_TRACK} bibliotekos funkcija
\emph{import()}.

Pikų skaičių mėginiuose atvaizduojanti stulpelinė diagrama sukurta su
\emph{ggplot2}\cite{R_GGPLOT} bibliotekos \emph{geom\_bar()} funkcija.

\subsection{Pikų skaičiaus nustatymas chromosomose}
Pikų skaičius skirtingose chromosomose kiekvienam
mėginiui apskaičiuotas, naudojantis standartine R funkcija \emph{length()},
pritaikyta atskiroms chromosomoms, kurių pozicijos aprašytos \emph{GRanges}
objekte. Gauti rezultatai normalizuoti apskaičiuotą pikų skaičių padalinus
iš kiekvienos chromosomos ilgio. Normalizavimas taikytas, siekiant įvertinti
pikų skaičių tarp chromosomų nepriklausomai nuo chromosomos ilgio.

Kiekvieno mėginio pikų pasiskirstymas
chromosomose atvaizduotas su \emph{ggplot()} ir papildoma funkcija
\emph{facet\_wrap()}, sukuriančia atskirus grafikus pagal pasirinktą elementą -
chromosomas.

\subsection{Persidengiančių pikų procentinė dalis}
Persidengiančių pikų tarp mėginių procentinė dalis nustatyta su modifikuota
\emph{Jaccard()} funkcija, apskaičiuojančia, kiek yra sutampančių regionų tarp
dviejų mėginių poros. Naudojantis nemodifikuota funkcija, \emph{Jaccard}
koeficientas apskaičiuojamas pagal išraišką:

                \[ J(A, B) =  \frac{|A \cap B|}{|A \cup B|} \]

\emph{Jaccard} koeficientas gaunamas iš rinkiniuose A ir B sutampančių duomenų
ilgio padalinus dviejų duomenų rinkinių bendrą duomenų ilgį.

\newpage

Modifikavus \emph{Jaccard} koeficiento gavimo funkciją, koeficientas
apskaičiuojamas pagal išraišką, kur sutampančių A ir B rinkinių duomenų ilgis
padalinamas iš A rinkinio ilgio:

                    \[ J(A, B) = \frac{|A \cap B|}{|A|} \]

\emph{Jaccard} koeficiento skaičiavimo funkcija modifikuota, nes skaičiuojant
koeficientą su standartine \emph{Jaccard} funkcija, gaunamas itin didelis
pikų sąjungos skaičius, o persidengiančių pikų skaičius - mažas,
todėl persidengiančių pikų skaičių padalinus iš pikų sąjungos gaunamas
itin mažas koeficientas, kurio apskaičiuota procentinė dalis neretai neviršijo
1\%.

Tam, jog būtų galima įvertinti, kokia pirmojo mėginio dalis persidengia su
antruoju mėginiu, persidengiančių pikų skaičius padalintas iš pirmojo
mėginio bendro pikų skaičiaus.

Gauti rezultatai atvaizduoti spalvų intensyvumo grafike (angl. \emph{heatmap}),
sukurtame su \emph{ggplot()} ir papildoma funkcija \emph{geom\_tile()}.

\subsection{Tbx5 motyvo nustatymas}
Šiame etape genomines pozicijas aprašantys bigBed formato failai
konvertuoti į BED formato failus, pasinaudojus UCSC komandinės eilutės programa
\emph{bigBedToBed}\cite{BBTOBED}.

Sugeneruoti BED formato failai panaudoti pikus atitinkančių sekų iš naminės
pelės genomo gavimui FASTA formatu. Sekos iš genomo išgautos, pasinaudojus
komandinės eilutės įrankio BEDTools\cite{BEDTOOLS} (2.30.0 versija) programa
\emph{getfasta}\cite{GET_FASTA}.

Kiekviename mėginyje esančio Tbx5 transkripcijos faktoriaus motyvo procentinė
dalis apskaičiuota susumavus \emph{Biostrings}\cite{BIOSTR} bibliotekos
funkcijos \emph{countPWM()} rezultatus. Funkcijai \emph{countPWM()} kaip
argumentas pateikta Tbx5 transkripcijos faktoriaus pozicinė svorių matrica bei
mėginio nukleotidų sekų rinkinys FASTA formatu. Gauta funkcijos reikšmė
padalinta iš bendro pikų skaičiaus.

Pozicinė svorių matrica atsisiųsta iš HOCOMOCO\cite{HOCOMOCO} (11.0 versija)
(angl. \emph{HOmo sapiens COmprehensive MOdel COllection}) duomenų bazės
\emph{Homo sapiens} ir \emph{Mus musculus} organizmų transkripcijos faktorių
kolekcijos. Pozicinę svorių matricą atitinkantis sekos logotipas vaizduojamas
pirmame paveiksle (1 pav.).

\begin{figure}[htb]
    \begin{center}
        \includegraphics[width=0.4\linewidth]{../Figures/tbx5_motif.png}
        \vspace{-2\baselineskip}
        \caption*{\small\textbf{1 pav. Tbx5 transkripcijos faktoriaus
                                sekos logotipas}}
    \end{center}
\end{figure}

Identifikuotų Tbx5 transkripcijos faktoriaus motyvų skaičius vizualizuotas su
pagrindinėmis \emph{ggplot()} ir \emph{geom\_bar()} funkcijomis.

\subsection{Motyvų paieška \emph{de novo}}
Praturtintų sekų radimui panaudota komandinės eilutės įrankio HOMER\cite{HOMER}
(v4.11 versija) programa \emph{findMotifsGenome.pl}, analizuojanti BED formato
failus (faile specifikuotas pozicijas), ir ieškanti praturtintų sekų atitikimo
anotuotame naminės pelės \emph{mm10} referentiniame genome.

\subsection{Praturtintų sekų biologinių funkcijų nustatymas}
Identifikuotų motyvų biologinės funkcijos nustatytos, pasinaudojus
UniProtKB\cite{UNIPROT} duomenų bazės genų ontologijos (angl. \emph{Gene
Ontology (GO)}) biologinių procesų, ląstelinių komponentų ir molekulinių
funkcijų klasifikacija.

\subsection{Analizės eigos schema}
Antrajame paveiksle (2 pav.) vaizduojama analizės etapus apibendrinanti schema:
\begin{figure}[htb]
    \begin{center}
        \includegraphics[width=1\linewidth]{../Figures/analysis_scheme.png}
        \vspace{-2\baselineskip}
        \caption*{\small\textbf{2 pav. Analizės atlikimo eiga}}
    \end{center}
\end{figure}

\newpage

%%%%%%%%%%%%%%%%%%%%%%%%%%%%%
% GAUTŲ REZULTATŲ APŽVALGA
%%%%%%%%%%%%%%%%%%%%%%%%%%%%%

%%%%%%%%%%%%%%%%%%%%%%%%%%%%%%%%
% REGIONŲ SKAIČIUS MĖGINIUOSE
%%%%%%%%%%%%%%%%%%%%%%%%%%%%%%%%

\section{Rezultatai ir jų aptarimas}
\subsection{Pikų skaičiaus skirtumai tarp mėginių}
Prieš pradedant Tbx5 motyvo paiešką, visiems mėginiams taikytas statistinis
duomenų vizualizavimas, siekiant įvertinti ChIP sekoskaitos duomenis bei
nustatyti, kaip skiriasi mėginiai, kuriuose tirtoms panašioms ląstelėms
taikyti skirtingi poveikiai.
Trečiame paveiksle (3 pav.) pateikiamoje stulpelinėje diagramoje vaizduojamas
pikų skaičiaus pasiskirstymas skirtinguose mėginiuose.

\begin{figure}[htb]
    \begin{center}
        \includegraphics[width=0.5\linewidth]{../Figures/total_peak_counts.png}
        \vspace{-2\baselineskip}
        \caption*{\small\textbf{3 pav. Pikų skaičių mėginiuose vaizduojanti
                                stulpelinė diagrama}}
    \end{center}
\end{figure}

Remiantis diagrama didžiausias pikų skaičius
nustatytas mėginyje \small\emph{emb\_fibr\_r1\_1}, kuriame pelių embrionų
fibroblastų ląstelės veiktos AGHMT faktoriais.
Šį rezultatą palyginus su kitais to paties eksperimento mėginiais, kuriuose
tirtas tas pats pelių embrionų fibroblastų ląstelių kamienas, tačiau ląstelės
veiktos tik kai kuriais faktoriais, pastebimas gradualus pikų skaičiaus
mažėjimas diagramoje
\small\emph{emb\_fibr\_r2\_2}, \small\emph{emb\_fibr\_r3\_3} ir
\small\emph{emb\_fibr\_r4\_4} pavaizduotuose stulpeliuose (pažymėta
oranžine spalva). Tarp šių mėginių pikų skirtumai ypač ryškūs - pikų
skaičius skiriasi \(\sim\)25 tūkst. ir \(\sim\)100 tūkst. pikų.

Mažiausiai pikų nustatyta mėginyje, kuriame tirti pelių naujagimių širdies
fibroblastai paveikti inhibitoriais. Nepaisant to, kad naudoti inhibitoriai
skatina širdies ląstelių diferenciaciją\cite{HEART_CELL_DIFF_ARTCL} ir darant
prielaidą, kad ChIP sekoskaitos duomenys yra tikslūs ir juose nėra
metodo klaidų, mažas pikų skaičius rodo, kad papildomas veikimas inhibitoriais
gali daryti mažą įtaką transkripcijos faktoriaus jungimuisi prie DNR sekų.

Remiantis gautais rezultatais pikų skaičius skirtingais poveikiais veiktose
panašiose ląstelėse varijuoja itin stipriai, tačiau šiuos rezultatus reikia
vertinti atsargiai, atsižvelgiant į galimus ChIP sekoskaitos metodu gautų
duomenų paruošimo netikslumus.

%%%%%%%%%%%%%%%%%%%%%%%%%%%%%%%%%%%%%%%%%%%%%
% PIKŲ SKAIČIUS ATSKIROSE CHROMOSOMOSE
%%%%%%%%%%%%%%%%%%%%%%%%%%%%%%%%%%%%%%%%%%%%%

\subsection{Pikų pasiskirstymas chromosomose}
Antrajame statistinių duomenų vaizdavimo etape patikrinta, kaip mėginių pikai
pasiskirstę atskirose chromosomose, atlikus normalizavimą pagal chromosomų ilgį.

Vaizduojamuose grafikuose (4 pav.) didžiausias pikų skaičius po
normalizavimo nustatytas penktoje, vienuoliktoje ir devynioliktoje
chromosomose. Šiose chromosomose didžiausią pikų skaičių turėjo mėginys,
kuriame tirtos iš naminių pelių širdies raumens išskirtos ląstelės, bei
mėginys, kuriame embrionų fibroblastai veikti AGHMT faktorių komplektu.
Taip pat galima pastebėti, kad pastarojo mėginio pikų skaičius beveik
visose chromosomose yra daug didesnis nei kituose mėginiuose apskaičiuotas
pikų skaičius.

Mažiausias pikų skaičius nustatytas lytinėse - X ir Y - chromosomose.

\begin{figure}[htb]
    \begin{center}
        \includegraphics[width=1\linewidth]{../Figures/peak_counts_by_chr.png}
        \vspace{-2\baselineskip}
        \caption*{\small\textbf{4 pav. Pikų pasiskirstymo chromosomose
                  stulpelinės diagramos}}
    \end{center}
\end{figure}

Iš visų chromosomų išsiskiria vienuolikta chromosoma. Remiantis MGI\cite{MGI}
(angl. \emph{Mouse Genome Informatics})
svetainės duomenimis, vienuoliktos \emph{Mus musculus} chromosomos
dydis - 122 megabazės, todėl gautas didžiausias pikų skaičius šioje
chromosomoje gali indikuoti, kad prie šioje chromosomoje esančių nukleotidų
sekų transkripcijos faktoriai jungiasi dažniau nei kitose chromosomose.

Remiantis pavaizduotomis pikų skaičiaus pasiskirstymo chromosomose
stulpelinėmis diagramomis, nustatyta, kad kai kuriose chromosomose mėginių
pikų skaičius gali būti mažesnis nei kitų mėginių, tačiau kitose chromosomose
tų pačių mėginių pikų skaičius gali būti pakankamai aukštas. Nepaisant to,
itin išsiskiriantis atrankumas chromosomų atžvilgiu nenustatytas, todėl galima
teigti, jog šiame analizės etape duomenų problematiškumas nepastebimas arba
jo nėra.

\newpage

%%%%%%%%%%%%%%%%%%%%%%%%%%%%%%%%%%%%%%%%%%%
% TARP MĖGINIŲ PERSIDENGIANTYS REGIONAI
%%%%%%%%%%%%%%%%%%%%%%%%%%%%%%%%%%%%%%%%%%%

\subsection{Tarp mėginių persidengiantys pikai}
Paskutiniame statistinių duomenų vaizdavimo etape nustatyta persidengiančių
mėginių duomenų procentinė dalis, siekiant įvertinti analizuojamų mėginių
panašumą.

\begin{figure}[htb]
    \begin{center}
        \includegraphics[width=0.6\linewidth]{../Figures/peak_overlaps.png}
        \vspace{-2\baselineskip}
        \caption*{\small\textbf{5 pav. Persidengiančių pikų procentinės
                                dalies spalvų intensyvumo grafikas}}
    \end{center}
\end{figure}

Remiantis pavaizduoto spalvų intensyvumo grafiko (5 pav.) duomenimis, didžiausi
persidengiančių pikų procentai nustatyti tarp šių mėginių:

\begin{itemize}
    \item \textbf{81.577 \%} - tarp mėginio, kuriame buvo tiriamos širdies
        fibroblastų ląstelės, veiktos inhibitoriais, ir mėginio, kuriame
        tirti embrionų fibroblastai, veikiant AGHMT.
    \item \textbf{81.092 \%} - tarp mėginio, kuriame tirti embrionų
        fibroblastai veikti AGHMT ir mėginio, kuriame nebuvo AKT1.
    \item \textbf{77.827 \%} - tarp mėginio su inhibitoriais veiktomis
        širdies fibroblastų ląstelėmis ir mėginio, kuriame nebuvo AKT1.
    \item \textbf{76.948 \%} - tarp mėginio, kuriame nebuvo HAND2 faktoriaus,
        ir mėginio, kuriame nebuvo AKT1.
    \item \textbf{75.439 \%} - tarp mėginio su inhibitoriais veiktomis
        širdies fibroblastų ląstelėmis ir tarp mėginio, kuriame nebuvo HAND2
        faktoriaus.
  \end{itemize}

Mažiausią panašumą su kitais mėginiais turi mėginys
(\small\emph{emb\_fibr\_r4\_4}), kuriame embrionų fibroblastų ląstelės
veiktos ne pilnu kardiogeninių transkripcijos faktorių (GATA4, HAND2, MEF2C,
Tbx5) komplektu.

Patikrinus, koks yra mėginių
(\small\emph{crdc\_mscl\_r1\_2} ir \small\emph{emb\_fibr\_r1\_1}),
kurie 4.2 poskyryje aprašytame analizės etape vienuoliktoje chromosomoje
turėjo didžiausią pikų skaičių, modifikuotas \emph{Jaccard} koeficientas,
buvo nustatyta, jog nors šie mėginiai turi daug pikų, tačiau tik 30.667\% ir
36.321\% šių pikų persidengia mėginių tarpusavio atžvilgiu, todėl tai rodo,
kad transkripcijos faktorių jungimasis prie chromatino yra nustatomas
skirtingose vienuoliktos ir kitų chromosomų pozicijose.

Apibendrinus 4.1, 4.2 ir 4.3 poskyriuose atliktas statistinių duomenų analizes
galima teigti, kad statistinis duomenų vaizdavimas, taikant skirtingas
vizualizavimo funkcijas bei metodus, padeda įvertinti duomenų skirtumus ir
grafikuose išskirti duomenų rinkinius, turinčius maksimalias ir minimalias
tiriamų dydžių reikšmes. Gautų analizių rezultatai parodė, kad ChIP sekoskaitos
duomenys skirtingais poveikiais veiktose ląstelėse skiriasi itin stipriai.
Taip pat, gauti rezultatai suteikė vertingos informacijos, kuri gali tapti
naujų analizių vykdymo priežastimi (pvz., išsiaiškinti, kodėl vienuoliktoje
naminės pelės chromosomoje DNR regionų, prie kurių jungiasi transkripcijos
faktoriai, yra daugiau nei kitose chromosomose).

%%%%%%%%%%%%%%%%%%%%%%%%%%%
% TBX5 MOTYVO NUSTATYMAS
%%%%%%%%%%%%%%%%%%%%%%%%%%%

\subsection{Tbx5 motyvo pasiskirstymas mėginiuose}
Įvertinus vizualizuotų duomenų skirtumus, vykdomos analizės metu nustatytas
Tbx5 motyvo taikinių skaičius skirtinguose mėginiuose. Rezultatas pateiktas
šeštame paveiksle (6 pav.).

\begin{figure}[htb]
    \begin{center}
        \includegraphics[width=0.5\linewidth]{../Figures/tf_hit_percentage.png}
        \vspace{-2\baselineskip}
        \caption*{\small\textbf{6 pav. Tbx5 motyvų taikinių skaičiaus palyginimo
                                sudėtinė diagrama}}
    \end{center}
\end{figure}

Daugiausiai Tbx5 motyvo sekos taikinių nustatyta mėginyje,
kuriame embrionų fibroblastai veikti serino/treonino kinaze 1 (Akt1) bei
keliais transkripcijos faktoriais (GATA4, HAND2, MEF2C, Tbx5)
(\small\emph{emb\_fibr\_r1\_1}).

Mažiausias Tbx5 motyvo sekų skaičius nustatytas mėginyje, kuriame embrionų
fibroblastai veikti tik vienu transkripcijos faktoriumi
(\small\emph{emb\_fibr\_r4\_4}), ir mėginyje, kuriame pelių naujagimių
širdies fibroblastai veikti inhibitoriais
(\small\emph{crdc\_fibr\_r1\_3}).

Remiantis sudėtine diagrama galima teigti, kad didesnis mėginių pikų skaičius
ne visada lemia didesnį transkripcijos faktoriaus motyvų skaičių. Tai įrodo
mėginyje \small\emph{crdc\_mscl\_r1\_2} nustatytas Tbx5 motyvo skaičius,
kurio procentinė dalis (8.32\%) yra mažesnė už mėginyje
\small\emph{crdc\_fibr\_r1\_3} nustatyto motyvo sekų skaičių (11.98\%),
nors \small\emph{crdc\_mscl\_r1\_2} mėginiui būdingas didesnis bendras pikų
skaičius.

\subsection{\emph{De novo} identifikuoti motyvai}
\emph{De novo} motyvų paieškos programos vykdymas buvo ilgiausiai trukęs
analizės etapas, lyginant su kitais analizės etapais. Šio etapo metu buvo
sugeneruoti HTML formato failai, kuriuose buvo pateiktas identifikuotų motyvų
sąrašas, išrikiuotas pagal \emph{p}-vertę didėjančia tvarka, motyvų sekų
logotipai bei nuorodos į puslapius su pozicinėmis motyvų svorių matricomis.

Antroje lentelėje (2 lentelė) kiekvienam mėginiui pateikti trys identifikuoti
motyvai, turintys mažiausią \emph{p}-vertę. Ketvirtame antros lentelės
stulpelyje nurodyta, kokią procentinę dalį mėginyje sudaro identifikuotas
Tbx5 motyvas.

\begin{table}[htb]
    \newcolumntype{M}[1]{>{\centering\arraybackslash}m{#1}}
    \small
    \caption*{\small\textbf{2 lentelė. Identifikuotų motyvų pavyzdžiai}}
    \vspace{0.1\baselineskip}
    \begin{tabular}{|c|c|c|c|c|}
    \hline
    \textbf{Mėginys} & \textbf{Pavadinimas} &
                       \textbf{\thead{Procentinė\\ dalis}} &
                       \textbf{Tbx5 motyvas} \\
    \hlineB{2.5}
    \multirow{3}{*}{\textbf{\emph{crdc\_mscl\_r1\_2*}}} & Tbx6(T-box)  & 20.28\% &
                                                \multirow{3}{*}{54.87\%} \\
    \cline{2-3}                                 & Tbet(T-box)  & 16.48\% & \\
    \cline{2-3}                                 & Eomes(T-box) & 25.49\% & \\
    \hlineB{2.5}
    \multirow{3}{*}{\textbf{\emph{emb\_fibr\_r1\_4*}}} & Mef2b(MADS) & 15.60\% &
                                               \multirow{3}{*}{44.22\%} \\
    \cline{2-3}                                & TRPS1(Zf) & 31.99\% & \\
    \cline{2-3}                                & GATA3(Zf) & 25.49\% & \\
    \hlineB{2.5}
    \multirow{3}{*}{\textbf{\emph{emb\_fibr\_r2\_4*}}} & Fos(bZIP) & 13.22\% &
                                               \multirow{3}{*}{42.97\%} \\
    \cline{2-3}                                & Fra1(bZIP) & 12.66\% & \\
    \cline{2-3}                                & Fra2(bZIP) & 11.36\% & \\
    \hlineB{2.5}
    \multirow{3}{*}{\textbf{\emph{emb\_fibr\_r3\_4*}}} & GATA3(Zf) & 25.06\% &
                                               \multirow{3}{*}{41.04\%} \\
    \cline{2-3}                                & TRPS1(Zf) & 31.55\% & \\
    \cline{2-3}                                & Fos(bZIP) & 11.66\% & \\
    \hlineB{2.5}
    \multirow{3}{*}{\textbf{\emph{heart\_r1\_1}}} & Mef2c(MADS) & 8.45\% &
                                           \multirow{3}{*}{39.08\%} \\
    \cline{2-3}                            & Mef2b(MADS) & 12.63\% & \\
    \cline{2-3}                            & Mef2d(MADS) & 5.42\% & \\
    \hlineB{2.5}
    \multirow{3}{*}{\textbf{\emph{emb\_fibr\_r4\_4*}}} & TRPS1(Zf) & 63.53\% &
                                               \multirow{3}{*}{26.57\%} \\
    \cline{2-3}                                & GATA3(Zf) & 52.43\% & \\
    \cline{2-3}                                & GATA4(Zf) & 38.94\% & \\
    \hlineB{2.5}
    \multirow{3}{*}{\textbf{\emph{crdc\_fibr\_r1\_3*}}} & Tbx6(T-box) & 38.28\% &
                                                \multirow{3}{*}{69.71\%} \\
    \cline{2-3}                                 & Tbet(T-box) & 33.27\% & \\
    \cline{2-3}                                 & Tbx21(T-box) & 30.60\% & \\
    \hline
    \end{tabular}
\end{table}

\let\thefootnote\relax\footnotetext{\textbf{*} - identifikuotas Tbx5 motyvas
statistiškai patikimas.}

Remiantis pateiktos lentelės duomenimis dažniausias mėginiuose pasikartojantis
motyvas priklauso TRPS1 ir GATA3 transkripcijos faktoriams. Didžiausia
Tbx5 motyvo procentinė dalis, atlikus \emph{de novo} motyvų paiešką su HOMER
įrankiu, nustatyta tarp širdies raumens ląstelių mėginio bei širdies
fibroblastų, veiktų inhibitoriais, mėginio (\small\emph{crdc\_mscl\_r1\_2},
\small\emph{crdc\_fibr\_r1\_3}).

Mėginyje, kuriame tirtos iš naminių pelių širdžių prieširdžių išgautos ląstelės
(\small\emph{heart\_r1\_1}), identifikuotas Tbx5 motyvas nėra statistiškai
patikimas, atsižvelgus į itin didelę \emph{p}-vertę. 

Pateiktoje stulpelinėje diagramoje (7 pav.) vaizduojamas bendras identifikuotų
motyvų skaičius kiekvienam mėginiui.

\begin{figure}[htb]
    \begin{center}
        \includegraphics[width=0.6\linewidth]{../Figures/motifs_in_samples.png}
        \vspace{-2\baselineskip}
        \caption*{\small\textbf{7 pav. Identifikuotų motyvų skaičiaus
                                mėginiuose stulpelinė diagrama}}
    \end{center}
\end{figure}

Nėra neįprasta, kad paskutinių mėginių (\small\emph{emb\_fibr\_r4\_4} ir
\small\emph{crdc\_fibr\_r1\_3}) identifikuotų motyvų skaičius yra pats
mažiausias - remiantis ankstesnių analizės etapų rezultatais (4.1, 4.2, 4.3
poskyriai) šie mėginiai turi mažiausią pikų skaičių bei mažiausią juos
atitinkančių sekų rinkinį.

Nepaisant šio atitikimo, remiantis 4.1 poskyrio rezultatais mėginys
\small\emph{heart\_r1\_1} turi \(\sim\) 50 tūkst. daugiau pikų nei
\small\emph{emb\_fibr\_r4\_4} ir jo bendras pikų skaičius nėra maksimalus,
tačiau \emph{de novo} motyvų identifikacijos etape šiame mėginyje buvo
nustatytas didžiausias motyvų skaičius (345 motyvai). Nustatyti motyvai yra
\(\sim\)12 nukleotidų ilgio.

Patikrinus, kiek vienodų motyvų aptinkama visuose mėginiuose, buvo nustatyta,
kad 131 motyvas yra būdingas visiems analizuojamiems mėginiams.

\newpage

\subsection{\emph{De novo} nustatytų motyvų biologinės funkcijos}
Identifikavus motyvus, buvo nustatytos su šiais motyvais susijusių genų
funkcijos. Remiantis atliktos paieškos UniProtKB\cite{UNIPROTKB} duomenų
bazėje rezultatais, trečioje lentelėje (3 lentelė) pateikti motyvai:

\begin{enumerate}
    \item \textbf{Tbx6:} Dalyvauja ląstelių proliferacijos ir organizacijos
        procesuose, signalinių kelių valdyme, kardioblastų diferenciacijoje.
    \item \textbf{Tbet:} Dalyvauja imuninės sistemos, baltymų metabolinių
        kelių procesuose. Taip pat pasireiškia organizmui reaguojant į
        dirgiklius.
    \item \textbf{Eomes:} Dalyvauja įvairių ląstelių (pvz.,
        kardiomiocitų) diferenciacijoje, neurogenezėje, kamieninių ląstelių
        populiacijos palaikyme.
    \item \textbf{Mef2b:} Dalyvauja įvairių ląstelių diferenciacijoje.
    \item \textbf{Mef2c:} Dalyvauja ląstelių apoptozėje, kraujagyslių
        formavimęsi, pradinės embrionų širdies vystymęsi.
    \item \textbf{Mef2d:} Dalyvauja suaugusių organizmų širdies vystymęsi,
        kremzlinių bei kaulinių ląstelių diferenciacijoje.
    \item \textbf{TRPS1:} Dalyvauja kremzlinių ląstelių diferenciacijoje,
        skeleto vystymęsi, būdingas neigiamas transkripcijos reguliavimas.
    \item \textbf{GATA3:} Dalyvauja aortos vožtuvų formavimęsi, širdies
        prieširdžių morfogenezėje, embrionų organų vystymęsi, eritrocitų
        diferenciacijoje.
    \item \textbf{GATA4:} Dalyvauja širdies ląstelių diferenciacijoje,
        širdies raumens regeneracijoje, embrionų širdies formavimęsi.
    \item \textbf{Fos:} Dalyvauja atsako į jonus (kadmio, kalcio),
        citokinus bei progesteroną procesuose. Taip pat aktyvus nervų sistemos
        vystymosi metu.
    \item \textbf{Fra1:} Dalyvauja ląstelės ciklo valdyme, apoptoziniuose
        bei embrionų vystymosi gimdoje procesuose.
    \item \textbf{Fra2:} Dalyvauja teigiamoje fibroblastų proliferacijoje,
        atsako į estradiolio - moteriško lytinio hormono - procesuose.
\end{enumerate}

Apibendrinus gautus biologinių funkcijų rezultatus, identifikuoti motyvai
atlieka svarbią funkciją organizmų vystymosi bei įvairių organų formavimosi
eigoje. Ypač svarbūs transkripcijos faktoriai, kurie dalyvauja ląstelių
diferenciacijoje (pvz., kardioblastų) bei širdies dalių formavimęsi, nes šių
transkripcijos faktorių panaudojimas gali būti itin reikšmingas
medicininiuose taikymuose.

\newpage

%%%%%%%%%%%%
% IŠVADOS
%%%%%%%%%%%%

\section{Išvados}
Atlikus Tbx5 transkripcijos faktoriaus analizę su skirtingais naminės pelės
ląstelių mėginiais, kuriems buvo taikyti skirtingi poveikiai, gauti rezultatai
buvo apibendrinti:

\begin{itemize}
    \item Iš duomenų bazės atsirinkti tinkamiausi mėginiai su panašiomis
        širdies audinių ląstelėmis, tačiau skirtingais taikytais poveikiais.
    \item Nepaisant panašių ląstelių analizavimo, skirtingų modifikacijų
        pritaikymas bei galimas netikslus ChIP sekoskaitos duomenų paruošimas
        lemia akivaizdžius duomenų skirtumus.
    \item Tbx5 transkripcijos faktoriaus motyvo taikinių skaičius tarp
        skirtingais poveikiais veiktų ląstelių mėginių skiriasi itin stipriai.
    \item \emph{De novo} identifikuoti motyvai yra susiję su ląstelių
        diferenciacija bei audinių regeneracija organizmo vystymosi eigoje.
\end{itemize}

\newpage

%%%%%%%%%%%%%%%%%%%%%%%%%%
% LITERATŪROS ŠALTINIAI
%%%%%%%%%%%%%%%%%%%%%%%%%%

\bibliographystyle{plain}
\begin{thebibliography}{99}

\bibitem{REGENERATION} Kate MacCord, Jane Maienschein (2019) Philosophy of
Biology: Understanding regeneration at different scales eLife 8:e46569
https://doi.org/10.7554/eLife.46569.

\bibitem{ORGANISMS} Mehta AS, Singh A. Insights into regeneration tool box:
An animal model approach. Dev Biol. 2019 Sep 15;453(2):111-129.
doi: 10.1016/j.ydbio.2019.04.006. Epub 2019 Apr 13. PMID: 30986388;
PMCID: PMC6684456.

\bibitem{GTRD} GTRD: an integrated view of transcription regulation.
Kolmykov S, Yevshin I, Kulyashov M, Sharipov R, Kondrakhin Y, Makeev VJ,
Kulakovskiy IV, Kel A, Kolpakov F Nucleic Acids Res. 2021 Jan
8;49(D1):D104-D111.

\bibitem{UCSCGB} UCSC Genome Browser: Kent WJ, Sugnet CW, Furey TS, Roskin KM,
Pringle TH, Zahler AM, Haussler D. The human genome browser at UCSC. Genome Res.
2002 Jun;12(6):996-1006.

\bibitem{ARTCL1} He A, Kong SW, Ma Q, Pu WT. Co-occupancy by multiple cardiac
transcription factors identifies transcriptional enhancers active in heart.
Proc Natl Acad Sci U S A. 2011 Apr 5;108(14):5632-7.
doi: 10.1073/pnas.1016959108. Epub 2011 Mar 17. PMID: 21415370;
PMCID: PMC3078411.

\bibitem{ARTCL2} Hashimoto H, Wang Z, Garry GA, Malladi VS, Botten GA, Ye W,
Zhou H, Osterwalder M, Dickel DE, Visel A, Liu N, Bassel-Duby R, Olson EN.
Cardiac Reprogramming Factors Synergistically Activate Genome-wide Cardiogenic
Stage-Specific Enhancers. Cell Stem Cell. 2019 Jul 3;25(1):69-86.e5.
doi: 10.1016/j.stem.2019.03.022. Epub 2019 May 9. PMID: 31080136;
PMCID: PMC6754266.

\bibitem{ARTCL3} Stone NR, Gifford CA, Thomas R, Pratt KJB, Samse-Knapp K,
Mohamed TMA, Radzinsky EM, Schricker A, Ye L, Yu P, van Bemmel JG, Ivey KN,
Pollard KS, Srivastava D. Context-Specific Transcription Factor Functions
Regulate Epigenomic and Transcriptional Dynamics during Cardiac Reprogramming.
Cell Stem Cell. 2019 Jul 3;25(1):87-102.e9. doi: 10.1016/j.stem.2019.06.012.
PMID: 31271750; PMCID: PMC6632093.

\bibitem{JCKSLAB} Jackson Laboratory (RRID:SCR\_004633).

\bibitem{R} R Core Team (2022).
R: A language and environment for statistical computing. R Foundation
for Statistical Computing, Vienna, Austria. URL https://www.R-project.org/.

\bibitem{SCIK}Scikick. Utility for executing collections of computational
notebooks.\\
URL https://petronislab.camh.ca/pub/scikick/stable/docs/report/out\_html/introduction.html.

\bibitem{R_TRACK} M. Lawrence, R. Gentleman, V. Carey: "rtracklayer: an {R}
package for interfacing with genome browsers". Bioinformatics 25:1841-1842.

\bibitem{R_GGPLOT} H. Wickham. ggplot2: Elegant Graphics for Data Analysis.
Springer-Verlag New York, 2016.

\bibitem{BBTOBED} BigWig and BigBed tools: Kent WJ, Zweig AS, Barber G,
Hinrichs AS, Karolchik D. BigWig and BigBed: enabling browsing of large
distributed data sets. Bioinformatics. 2010 Sep 1;26(17):2204-7.

\bibitem{BEDTOOLS} Quinlan AR, Hall IM. BEDTools: a flexible suite of
utilities for comparing genomic features. Bioinformatics. 2010 Mar
15;26(6):841-2. doi: 10.1093/bioinformatics/btq033. Epub 2010 Jan 28.
PMID: 20110278; PMCID: PMC2832824.

\bibitem{GET_FASTA} BEDTools komandinės eilutės įrankis. Programų rinkinio
programa \emph{getfasta}.\\
Prieiga per https://bedtools.readthedocs.io/en/latest/content/tools/getfasta.html
[žiūrėta 2022-06-03].

\bibitem{BIOSTR} Pagès H, Aboyoun P, Gentleman R, DebRoy S (2022). \_Biostrings:
Efficient manipulation of biological strings\_. R package version
2.64.0, <https://bioconductor.org/packages/Biostrings>.

\bibitem{HOCOMOCO} HOCOMOCO: towards a complete collection of transcription
factor binding models for human and mouse via large-scale ChIP-Seq analysis
Ivan V. Kulakovskiy; Ilya E. Vorontsov; Ivan S. Yevshin; Ruslan N. Sharipov;
Alla D. Fedorova; Eugene I. Rumynskiy; Yulia A. Medvedeva; Arturo Magana-Mora;
Vladimir B. Bajic; Dmitry A. Papatsenko; Fedor A. Kolpakov; Vsevolod J. Makeev
Nucl. Acids Res., Database issue, gkx1106 (11 November 2017)
doi: 10.1093/nar/gkx1106.

\bibitem{HOMER} Heinz S, Benner C, Spann N, Bertolino E et al. Simple
Combinations of Lineage-Determining Transcription Factors Prime cis-Regulatory
Elements Required for Macrophage and B Cell Identities. Mol Cell 2010 May
28;38(4):576-589. PMID: 20513432.

\bibitem{UPSETR} Jake R. Conway, Alexander Lex, Nils Gehlenborg. UpSetR: An R
Package For The Visualization Of Intersecting Sets And Their Properties
Bioinformatics, 33(18): 2938-2940, doi:10.1093/bioinformatics/btx364, 2017.

\bibitem{UNIPROT} The UniProt Consortium UniProt: the universal protein
knowledgebase in 2021 Nucleic Acids Res. 49:D1 (2021).

\bibitem{UNIPROTKB} Boutet E, Lieberherr D, Tognolli M, Schneider M, Bairoch A.
UniProtKB/Swiss-Prot Methods Mol. Biol. 406:89-112 (2007).

\bibitem{HEART_CELL_DIFF_ARTCL} Drowley L, Koonce C, Peel S, et al. Human
Induced Pluripotent Stem Cell-Derived Cardiac Progenitor Cells in Phenotypic
Screening: A Transforming Growth Factor-β Type 1 Receptor Kinase Inhibitor
Induces Efficient Cardiac Differentiation. Stem Cells Transl Med.
2016;5(2):164-174. doi:10.5966/sctm.2015-0114.

\bibitem{MGI} Bult CJ, Blake JA, Smith CL, Kadin JA, Richardson JE, the
Mouse Genome Database Group. 2019. Mouse Genome Database (MGD) 2019.
Nucleic Acids Res. 2019 Jan. 8;47 (D1): D801–D806.

\end{thebibliography}

\newpage

%%%%%%%%%%%%
% PRIEDAI
%%%%%%%%%%%%

\section{Priedas}

Priedų sąraše pateikiamos tarpinių rezultatų puslapio, sugeneruoto su Scikick,
bei Git repozitorijos, kurioje saugomi analizei naudoti duomenų failai,
parašyti skriptai bei pagrindinė R programa, nuorodos.

\begin{itemize}
    \item \textbf{Tarpinių rezultatų Scikick puslapis:}\\
        https://karklas.mif.vu.lt/~dast6577/KursinisDarbas/v2.0/peaks\_MM.html
    \item \textbf{Analizės Git repozitorija:}\\
        https://github.com/dansta0804/Tbx5\_analysis.git
  \end{itemize}

\end{document}
