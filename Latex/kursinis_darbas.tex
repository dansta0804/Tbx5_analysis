\documentclass[12pt]{article}

%\documentclass[17pt]{extarticle}

%\usepackage{extsizes}
\usepackage{indentfirst}
\usepackage[utf8x]{inputenc}
\usepackage[T1]{fontenc}
\usepackage[english,lithuanian]{babel}
\usepackage{array}
\usepackage{caption}

\usepackage{amsmath, amsthm, amssymb}
\usepackage{graphicx}
\usepackage{setspace}
\usepackage{verbatim}
\usepackage[left=3cm,top=2cm,right=1.5cm,bottom=2cm]{geometry}
 %\textwidth 6.5in
 %\textheight 9.00in

  \onehalfspacing

% Author's MACROS
 \newcommand{\EE}{\mathbb{E}\,} % Mean
 \newcommand{\ee}{{\mathrm e}}  % nice exponent
 \newcommand{\dd}{{\mathrm d}}
 \newcommand{\RR}{\mathbb{R}}
 %Macros end

\begin{document}
\selectlanguage{lithuanian}

\begin{titlepage}
\vskip 20pt
\begin{center}
\includegraphics[scale=0.5]{MIF}
\end{center}

%%%%%%%%%%%%%%%%%%%%%%%%%%%%%%%%%%%%%%%%
% TITULINIO PUSLAPIO TEKSTAS
%%%%%%%%%%%%%%%%%%%%%%%%%%%%%%%%%%%%%%%%

\vskip 20pt
\centerline{\bf \large \textbf{VILNIAUS UNIVERSITETAS}}
\bigskip
\centerline{\large \textbf{MATEMATIKOS IR INFORMATIKOS FAKULTETAS}}
\bigskip
\centerline{\large \textbf{BIOINFORMATIKOS BAKALAURO STUDIJŲ PROGRAMA}}

\vskip 90pt
\begin{center}
    {\bf \LARGE \emph{tbx5} ir \emph{tcf21} genų įtakos regeneracijai tyrimai}
\end{center}
\begin{center}
    {\bf \Large Research of the \emph{tbx5} and \emph{tcf21} genes' functions in regeneration}
\end{center}
\vskip 20pt
\centerline{\bf \Large \textbf{Kursinis darbas}}
\bigskip
\vskip 50pt

\hskip 140pt {\Large Autorius: Danielė Stasiūnaitė}

\hskip 140pt{\Large VU el. p.: (daniele.stasiunaite@mif.stud.vu.lt)}
\bigskip
\vskip 20pt

\hskip 140pt {\Large Darbo vadovas: J. Asist. Kotryna Kvederavičiūtė}
\vskip 60pt
\vskip 60pt
\centerline{\large \textbf{Vilnius}}
\centerline{\large \textbf{2022}}
\newpage
\end{titlepage}

\selectlanguage{lithuanian}

%%%%%%%%%%%%%%%%%%%%%%%%%%%%%%%%%%%%%%%%
% TURINIO PUSLAPIS
%%%%%%%%%%%%%%%%%%%%%%%%%%%%%%%%%%%%%%%%  
\tableofcontents
\newpage

%%%%%%%%%%%%%%%%%%%%%%%%%%%%%%%%%%%%%%%%
% LIETUVIŠKOS SANTRAUKOS PUSLAPIS
%%%%%%%%%%%%%%%%%%%%%%%%%%%%%%%%%%%%%%%%  
\section*{Santrauka}
Darbo santrauka.\\

\textbf{Raktiniai žodžiai: TF, ChIP-seq, regionas}
\newpage

%%%%%%%%%%%%%%%%%%%%%%%%%%%%%%%%%%%%%%%%
% ANGLIŠKOS SANTRAUKOS PUSLAPIS
%%%%%%%%%%%%%%%%%%%%%%%%%%%%%%%%%%%%%%%%
\section*{Summary}
Short summary of results.\\

\textbf{Keywords?}
\newpage

%%%%%%%%%%%%%%%%%%%%%%%%%%%%%%%%%%%%%%%%
% ĮVADO PUSLAPIS
%%%%%%%%%%%%%%%%%%%%%%%%%%%%%%%%%%%%%%%%
\section{Įvadas}
\newpage

%%%%%%%%%%%%%%%%%%%%%%%%%%%%%%%%%%%%%%%%
% DUOMENŲ APŽVALGA
%%%%%%%%%%%%%%%%%%%%%%%%%%%%%%%%%%%%%%%%
\section{Duomenys}

Tyrime naudoti duomenys atsisiųsti iš \emph{GTRD} (Gene Transcription
Regulation Database) duomenų bazės, saugančios informaciją apie įvairių
transkripcijos faktorių jungimosi prie DNR sekų regionus. Ši duomenų
bazė pasirinkta dėl sistemiškai surinktų ChIP-seq eksperimentų, kurių
metu gauti rezultatai unifikuotai apdorojami ir paruošiami tyrėjų
analizėms. Taip pat \emph{GTRD} duomenų bazėje duomenys saugomi
binariniu anotacijų formatu \emph{bigBed}, leidžiandžiu atvaizduoti
pasirinktą chromosomos regioną genomų naršyklėje sparčiau ir 
efektyviau nei tekstinis \emph{BED} formatas.

Žemiau esančioje lentelėje pateikta informacija apie tyrimui atlikti
naudotus duomenis, surinktus iš naminės pelės \emph{(lot. Mus musculus)}
ląstelių. Visi duomenys gauti iš eksperimentų, kurių metu buvo
siekiama išsiaiškinti \emph{tbx5} ir kitų transkripcijos faktorių
įtaką naujų širdies ląstelių, kardiomiocitų, susidarymui bei širdies
funkcijų atsistatymui po ištikusio infarkto arba kitų, su širdies
pažeidimais susijusių, sutrikimų.

\begin{table}[htb]
    \newcolumntype{M}[1]{>{\centering\arraybackslash}m{#1}}
    \small
    \caption*{1 lentelė. Mėginių charakteristikos}
    \begin{tabular}{|M{2cm}|M{3.5cm}|M{3.7cm}|M{3cm}|M{3cm}|M{0.5cm}|}
    \hline
    \textbf{GTRD ID} & \textbf{Ląstelių tipas} &
                    \textbf{Sąlygos pritaikymas} & \textbf{Antikūnai} &
                    \textbf{PubMed ID}\\
    \hline
    EXP030893 & HL - 1 (širdies raumens) & Kontrolė & - & 21415370\\
    \hline
    EXP058852 & Širdies prieširdžių & Kamienas: C57BL/6; vystymosi
                stadija: p4 & Tbx5 (sc-17866) & 31080136\\
    \hline
    EXP062056 & Pelių naujagimių širdies fibroblastų, ekspresuojančių didelį
                kiekį T antigeno, linija & Kamienas: CD1; sąlyga: sb431542,
                    xav939 & anti-TBX5 (sc-17866x) & 31271750\\
    \hline
    EXP058843 & MEF (embrionų fibroblastai) & Kamienas: C57BL/6; sąlyga: AGHMT
                (2 d.) & anti-Tbx5 (sc-17866) & 31080136\\
    \hline
    EXP058847 & MEF (embrionų fibroblastai) & Kamienas: C57BL/6; sąlyga: GHMT
                (2 d.) & Tbx5 (sc-17866) & 31080136\\
    \hline
    EXP058850 & MEF (embrionų fibroblastai) & Kamienas: C57BL/6; sąlyga: GMT
                (2 d.) & Tbx5 (sc-17866) & 31080136\\
    \hline
    EXP058856 & MEF (embrionų fibroblastai) & Kamienas: C57BL/6; sąlyga: vienas
                faktorius (2 d.) & Tbx5 (sc-17866) & 31080136\\
    \hline
    \end{tabular}
\end{table}

Tyrimui atlikti pasirinkti septynių eksperimentų duomenys tarp kurių buvo
tiriami keturi embrionų fibroblastų kamienai, kuriems dvi dienas taikytos
skirtingos sąlygos: veikimas AGHMT (AKT1 - serino/treonino kinazė 1; GATA4,
HAND2, MEF2C, TBX5 - kardiogeniniai transkripcijos faktoriai), GHMT, GMT
ir tik vienu transkripcijos faktoriumi, kuris nebuvo specifikuotas.

Taip pat tirti pelių naujagimių fibroblastai, veikti inhibitoriais: 
sb431542, skatinančiu kardiomiocitų diferenciaciją iš pliuripotentinių
kamieninių ląstelių, ir xav939, sukeliančiu progenitorinių ląstelių
kardiomiogenezę.

Į duomenų rinkinį įtraukti du mėginiai, kuriems nebuvo taikytos papildomos
sąlygos: kontrolinis HL - 1 širdies raumens ląstelių mėginys bei p4
vystymosi stadijos širdies prieširdžių ląstelių mėginys.
\newpage

%%%%%%%%%%%%%%%%%%%%%%%%%%%%%%%%%%%%%%%%
% METODAI
%%%%%%%%%%%%%%%%%%%%%%%%%%%%%%%%%%%%%%%%

\section{Tyrimo metodai}
\emph{Tbx5} transkripcijos faktoriaus regionų tyrimo analizė atlikta
su R programavimo kalba (4.2.0 versija). Žemiau esančiuose atskiruose
analizės etapų skyriuose nurodomos R bibliotekos ir papildomi komandinės
eilutės įrankiai, kuriais naudojantis pasiekti tarpiniai analizės etapų
rezultatai.

\subsection{\emph{Tbx5} regionų skaičiaus nustatymas mėginiuose}
\emph{Tbx5} regionų skaičius skirtinguose mėginiuose apskaičiuotas
\emph{rtracklayer} bibliotekos sukurtam GRanges objektui, aprašančiam
genomines pozicijas bei su jomis susijusias anotacijas, pritaikius
standartinę R ilgio funkciją \emph{length()}.
Regionų skaičių mėginiuose atvaizduojanti stulpelinė diagrama sukurta
su \emph{ggplot2} bibliotekos \emph{geom\_bar()} funkcija.

\subsection{\emph{Tbx5} regionų skaičiaus pasiskirstymas chromosomose}
Transkripcijos faktoriaus regionų skaičius skirtingose chromosomose
suskaičiuotas, naudojantis standartine R funkcija \emph{length()},
pritaikyta atskiroms chromosomoms, kurių pozicijos aprašytos GRanges
objekte.
Kiekvieno mėginio \emph{Tbx5} transkripcijos faktoriaus pasiskirstymas
chromosomose atvaizduotas su \emph{ggplot2} biblioteka, panaudojus
papildomą funkciją \emph{facet\_wrap()}, sukuriančia atskirus 
grafikus chromosomoms.

\subsection{Persidengiančių regionų mėginiuose analizė}
Persidengiančių regionų tarp mėginių procentinė dalis nustatyta
su \emph{Jaccard()} funkcija, apskaičiuojančia, kiek yra sutampančių
regionų tarp dviejų mėginių poros.
Gauti rezultatai atvaizduoti spalvų intensyvumo grafike \emph{angl.
heatmap}, sukurtame \emph{ggplot2} bibliotekos funkciją \emph{ggplot}
papildžius funkcija \emph{geom\_tile()}.

\subsection{\emph{Tbx5} motyvo nustatymas}
Šiame etape genomines pozicijas aprašantys \emph{bigBed} formato
failai konvertuoti į \emph{BED} formato failus, pasinaudojus
\emph{UCSC} komandinės eilutės programa \emph{bigBedToBed}\footnote[1]
{\emph{bigBedToBed} aprašymas: 
https://genome.ucsc.edu/goldenPath/help/bigBed.html}.
Sugeneruoti \emph{BED} formato failai panaudoti pikus atitinkančių
sekų iš naminės pelės \emph{Mus musculus} genomo gavimui
\emph{fasta} formatu. Sekos atsisiųstos, pasinaudojus komandinės
eilutės įrankio \emph{BEDTools}\footnote[2]{BEDTools paketo
dokumentacija: 
https://bedtools.readthedocs.io/en/latest/content/bedtools-suite.html}
programa \emph{getfasta}\footnote[3]{\emph{getfasta} funkcijos
dokumentacija: 
https://bedtools.readthedocs.io/en/latest/content/tools/getfasta.html}.

Kiekviename mėginyje esančio \emph{tbx5} transkripcijos faktoriaus
motyvo procentinė dalis apskaičiuota susumavus \emph{Biostrings}
bibliotekos funkcijos \emph{countPWM()} rezultatus bei gautą
vertę padalinus iš bendro regionų skaičiaus.
Rezultatai atvaizduoti, pasinaudojus \emph{ggplot} bibliotekos
funkcijomis.

\subsection{Motyvų paieška \emph{de novo}}
Tekstas.

\subsection{Sekų praturtinimo analizė}
Tekstas.

\newpage

%%%%%%%%%%%%%%%%%%%%%%%%%%%%%%%%%%%%%%%%
% GAUTŲ REZULTATŲ APŽVALGA
%%%%%%%%%%%%%%%%%%%%%%%%%%%%%%%%%%%%%%%%

%%%%%%%%%%%%%%%%%%%%%%%%%%%%%%%%%%%%%%%%
% REGIONŲ SKAIČIUS MĖGINIUOSE
%%%%%%%%%%%%%%%%%%%%%%%%%%%%%%%%%%%%%%%%
\section{Rezultatų apžvalga}
\subsection{Regionų skaičiaus skirtumai tarp mėginių}
Pirmajame analizės etape kiekviename mėginyje nustatytas bendras
regionų skaičius pavaizduotas žemiau esančioje stulpelinėje
diagramoje.

\begin{figure}[htb]
    \begin{center}
        \includegraphics[width=0.7\linewidth]{Figures/total_peak_counts.png}
        \caption*{1 pav. Regionų skaičių kiekviename mėginyje vaizduojanti diagrama}
    \end{center}
\end{figure}

Remiantis diagrama didžiausias \emph{tbx5} transkripcijos faktoriaus
regionų skaičius nustatytas eksperimento \emph{(mm\_4\_emb\_fibr\_r1)}
biologinėje replikoje, kurioje pelių embrionų fibroblastų ląstelės
dvi dienas veiktos AGHMT (AKT1, GATA4, HAND2, MEF2C ir TBX5) faktoriais.
Šį rezultatą palyginus su kitomis biologinėmis replikomis, kuriose
tirtas tas pats pelių embrionų fibroblastų ląstelių kamienas, tačiau
ląstelės veiktos tik kai kuriais faktoriais, pastebimas gradualus
\emph{tbx5} transkripcijos faktoriaus regionų skaičiaus mažėjimas
diagramoje \emph{mm\_4\_emb\_fibr\_r2}, \emph{mm\_4\_emb\_fibr\_r3} ir
\emph{mm\_4\_emb\_fibr\_r4} pavaizduotuose stulpeliuose. Mėginyje,
kuriame embrionų fibroblastai veikti tik vienu faktoriumi,
\emph{tbx5} transkripcijos faktoriaus regionų nustatoma nedaug -
13520.

Tai leidžia daryti išvadą, kad \emph{tbx5} transkripcijos faktoriaus
geno ekspresija priklauso nuo kardiogeninių faktorių buvimo.

Mažiausiai regionų nustatyta mėginyje, kuriame tirta pelių
naujagimių širdies fibroblastų, ekspresuojančių T antigeną
ir paveiktų inhibitoriais: \emph{sb431542} ir \emph{xav939}.
Nepaisant to, kad abu inhibitoriai skatina širdies ląstelių
diferenciaciją, itin mažas transkripcijos faktoriaus regionų
skaičius rodo, kad papildomas veikimas inhibitoriais daro
mažą įtaką transkripcijos faktoriaus pasireiškimui.

\newpage

%%%%%%%%%%%%%%%%%%%%%%%%%%%%%%%%%%%%%%%%%%%%%
% REGIONŲ SKAIČIUS ATSKIROSE CHROMOSOMOSE
%%%%%%%%%%%%%%%%%%%%%%%%%%%%%%%%%%%%%%%%%%%%%
\subsection{Regionų pasiskirstymas atskirose chromosomose}
Nustačius \emph{tbx5} transkripcijos faktoriaus regionų
pasiskirstymą eksperimentų mėginiuose, kitame analizės etape
patikrinta, kaip faktoriaus regionai pasiskirstę atskirose
chromosomose.

Vaizduojamuose grafikuose didžiausias regionų skaičius nustatomas
pirmoje, antroje ir penktoje chromosomose. Naminės pelės pirmoji
chromosoma yra pati didžiausia, turinti 195 milijonų bazių porų,
antroji chromosoma sudaryta iš 182 megabazių, penktoji chromosoma -
152 milijonų bazių porų, todėl didesnis regionų skaičius šiose
chromosomose yra įprastas. Kitose chromosomose regionų skaičius
yra mažesnis. Ypač mažas regionų skaičius nustatomas devynioliktoje
(61 Mbp), X (169 Mbp) ir Y (91 Mbp) chromosomose.

Biologinių replikų mėginiuose didžiausias regionų skaičius
nustatytas antroje chromosomoje. Taip pat grafikuose išsiskiria
kontrolinis HL - 1 širdies ląstelių mėginys, kuriame didžiausias
regionų skaičius nustatomas penktojoje chromosomoje.

\begin{figure}[htb]
    \begin{center}
        \includegraphics[width=0.8\linewidth]{Figures/peak_counts_by_chromosome.png}
        \caption*{2 pav. Regionų pasiskirstymas chromosomose}
    \end{center}
\end{figure}

\newpage

%%%%%%%%%%%%%%%%%%%%%%%%%%%%%%%%%%%%%%%%%%%%
% TARP MĖGINIŲ PERSIDENGIANTYS REGIONAI
%%%%%%%%%%%%%%%%%%%%%%%%%%%%%%%%%%%%%%%%%%%%


\begin{figure}[htb]
    \begin{center}
        \includegraphics[width=0.8\linewidth]{Figures/peak_overlaps_between_samples.png}
        \caption*{3 pav. Persidengiančių regionų procentinės dalies vaizdavimas}
    \end{center}
\end{figure}

\newpage

%%%%%%%%%%%%%%%%%%%%%%%%%%%%%%%%%%%%%%%%
% LITERATŪROS ŠALTINIAI
%%%%%%%%%%%%%%%%%%%%%%%%%%%%%%%%%%%%%%%%

\bibliographystyle{plain}
\begin{thebibliography}{99}

\bibitem{YLS16}1. Yang Yang; R. Leipus, J. \v{S}iaulys. Asymptotics for randomly weighted and stopped dependent sums, \emph{Stochastics: an international journal of probability and stochastic processes}, 2016, \textbf{88}(2), p.p. 300-319.
 \bibitem{E05} T. Erhardsson. Stein’s method for Poisson and compound Poisson approximation, \emph{In: An Introduction to Stein’s Method. Lect. Notes Ser. Inst. Math. Sci. Natl. Univ. Singap. v. 4}, Singapore: Singapore Univ. Press, 2005, p.p. 61–113.
\bibitem{KGDD08} R. Kaas, M. Goovaerts, J. Dhaene  and M. Denuit. \emph{Modern Actuarial Risk Theory:
using R.} (Secon. edt.), Springer-Verlag, Berlin, Heidelberg,  2008, 393 p.
\bibitem{St12} V. Stakėnas, \emph{Tikimybių teorija ir matematinė statistika} (rankraštis), 2012, 178 p. , http://www.statistika.mif.vu.lt/atsisiuntimui/statistika/

\end{thebibliography}
\newpage

%%%%%%%%%%%%%%%%%%%%%%%%%%%%%%%%%%%%%%%%
% PRIEDAI
%%%%%%%%%%%%%%%%%%%%%%%%%%%%%%%%%%%%%%%%

\section{Priedas}
Programiniam kodui pateikti galima naudoti  \verb|\verbatim| komandą:
\small  %mažesnės raidės
\begin{verbatim}

R arba Python programinis kodas.

\end{verbatim}



\section{Literatūros apžvalga}
Literatūros darbo tema apžvalga. Cituojant \cite{YLS16,KGDD08}. Nepamirškite pdflatex įvykdyti bent du kartus, kad pamatytumėte citavimą, ne klaustukus.


\end{document}
